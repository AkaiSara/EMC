\documentclass[12pt,a4paper]{report}
\usepackage[T1]{fontenc}
\usepackage[utf8]{inputenc}
\usepackage[italian]{babel}

\setcounter{secnumdepth}{2} %modifica numero livelli numerabili
\setcounter{tocdepth}{2}    %modifica numero livelli visualizzati nella tableofcontents(toc)

\usepackage{graphicx}
\usepackage{hyperref}
\hypersetup{hidelinks}

\title{Elementi di Marketing e Comunicazione \\
\small{Prof. Alessandra Falco}}

\author{Alessandro S.}

\begin{document}

\maketitle
\tableofcontents
\clearpage

\chapter{Introduzione}
\section{Nascita del Marketing}
\subsubsection*{Fase 1: Orientamento al prodotto: si vende cosa si può produrre}
Nella fase iniziale dello sviluppo industriale gli acquirenti erano ricettivi e poco esigenti e il mercato era caratterizzato da una scarsità di beni e servizi. L'attenzione delle aziende era quindi rivolta alle fasi tecnico-ingegneristiche di progettazione e fabbricazione del prodotto.

\subsubsection*{Fase 2: Orientamento alla vendita}
In questa fase si sviluppano i mercati di massa e l'offerta supera nel suo complesso la domanda. In tale contesto diventa critico il ruolo della funzione vendite rispetto alla fase precedente, in quanto si avverte la concorrenza delle altre aziende, si cerca quindi di allargare le proprie vendite sottraendo quote di mercato ai concorrenti. \newline
Si fa strada il convincimento che con adeguati investimenti in attività pubblicitarie e promozionali possa essere ampliata la domanda di qualsiasi prodotto. Ne conseguono strategie di vendita aggressive e di massa.

\subsubsection*{Fase 3: Orientamento al mercato: si produce cosa si può vendere}
La vendita "aggressiva" si dimostra insufficiente per sostenere il lancio di un prodotto non in sintonia con la domanda. L'attenzione, dunque, si sposta dalla vendita ai bisogni del consumatore. Bisogna orientare la produzione in funzione dei mutevoli bisogni dei consumatori spesso influenzati dalla moda.\newline
Con l'adozione dell'orientamento al mercato delle aziende nasce il marketing.

\section{Che cos'è il marketing}
\subsection{Definizione}
Il marketing rappresenta l’insieme delle attività mediante le quali un’organizzazione mira a soddisfare le esigenze di persone rendendo loro disponibili prodotti o servizi, sostenendo idee o affermando valori nella società.

\subsection{Bisogni VS Desideri}
Il bisogno è la mancanza totale o parziale di uno o più elementi che costituiscono il benessere della persona. I bisogni non sono creati dalla società o dal marketing; essi esistono nella sostanza stessa della biologia e della condizione umana. (vedere Piramide di Maslow).
Il desiderio è un sentimento intenso che spinge a cercare il possesso, il conseguimento o l’attuazione di quanto possa appagare un proprio bisogno. \newline
Mentre i bisogni sono relativamente pochi i desideri sono molteplici. I desideri umani vengono continuamente modellati e rimodellati. \newline
Il marketing, dunque, assieme ad altri agenti sociali, può solo influenzare e creare i desideri e non i bisogni.

\subsection{Cosa non è}
Il marketing non può essere considerato equivalente alla vendita in quanto ha inizio molto prima che un prodotto sia reso disponibile. Il marketing è l’attività che i manager devono svolgere al fine di identificare i bisogni attuali e potenziali, misurare la loro intensità e determinare se si presentano opportunità profittevoli. La vendita ha luogo solo dopo che un prodotto è stato realizzato. \newline
Il marketing non è una specifica funzione dell’impresa, è troppo importante per essere affidato esclusivamente alla funzione commerciale , ma è un orientamento culturale che guida l’insieme delle attività organizzative.

\subsection{Tipi di marketing}
\subsubsection*{Marketing dei beni e dei servizi} 
Riguarda gli scambi tra le imprese e i consumatori. Business To Consumer (B2C)
\subsubsection*{Marketing industriale} 
Riguarda gli scambi tra imprese e altre imprese. Business To Business (B2B)
\subsubsection*{Marketing sociale}
Il marketing sociale è l’insieme di attività intraprese da un’organizzazione per creare , mantenere e modificare gli atteggiamenti e i comportamenti collettivi in vista di valori socialmente condivisi. \newline
Il marketing sociale affronta temi sociali di carattere non controverso, quindi ad esempio la politica essendo controversa non è marketing sociale. \newline
Il marketing sociale spesso riguarda le organizzazioni senza fini di lucro. \newline
I principali settori in cui opera il marketing sociale riguardano
\begin{itemize}
	\item  la promozione e la tutela della salute (es. diffusione di abitudini alimentari e di stili di vita corretti)
	\item  la prevenzione e la riduzione di comportamenti a rischio (es. assunzione di fumo, alcool, droghe, comportamenti sessuali a rischio, comportamenti scorretti di guida, dipendenze comportamentali)
	\item la promozione di comportamenti sociali (es.risparmio energetico, riduzione inquinamento atmosferico, utilizzo del trasporto pubblico, raccolta  differenziata, donazione di sangue).
\end{itemize}

\subsection{Livelli di realizzazione del marketing}
\subsubsection*{Marketing di risposta}
Questa forma di marketing si manifesta quando esistono dei bisogni chiaramente definiti, con riferimento ai quali un certo numero di imprese si impegna a predisporre soluzioni valide.

\subsubsection*{Marketing di anticipo}
Questa forma di marketing riconosce un bisogno emergente o latente. È più rischioso del marketing di risposta: le imprese possono entrare sul mercato o troppo presto o troppo tardi, oppure possono commettere errori di valutazione ritenendo che un certo mercato possa svilupparsi.

\subsubsection*{Creazione di prodotti/servizi non richiesti}
Il marketing raggiunge il massimo livello quando un’impresa (imprese market-driving) introduce un prodotto o un servizio in precedenza non richiesto o immaginato da alcuno.

\subsection{Conoscenza degli altri operatori di mercato}
Conoscenza di influenzatori, intermediari e concorrenti.

\section{Opportunità di mercato}
\textit{Il marketing è l’arte di individuare, sviluppare e ricavare un profitto dalle opportunità. (Kotler, 1999) } \newline
Le situazioni che possono determinare il sorgere di opportunità di mercato sono principalmente tre:
\begin{itemize} 
	\item possibilità di fornire un prodotto o un servizio a offerta scarsa;
	\item possibilità di fornire un prodotto o un servizio esistente in modo nuovo o migliore;
	\item possibilità di fornire un prodotto o un servizio nuovi per il mercato.
\end{itemize}

Per capire verso cosa indirizzarsi si può chiedere agli utilizzatori di prodotti o servizi se riscontrano problemi nell’impiego dei prodotti in questione e se hanno dei suggerimenti da fare; oppure creare all’interno dell’azienda un sistema di raccolta che faccia confluire le nuove idee verso un punto centrale dove possono essere classificate, analizzate e valutate.

\section{Pensiero creativo}
Per progredire e quindi stimolare l’innovazione, non bastano i fondi, le risorse finanziarie, ma occorre, in pari misura, un’adeguata espressione e applicazione di quello che viene chiamato “pensiero divergente”.

\subsubsection*{Pensiero convergente}
Questo tipo di pensiero è caratterizzato dalla ripetizione del già appreso e dall’adattare vecchie risposte a situazioni nuove in modo più o meno meccanico.

\subsubsection*{Pensiero divergente}
Il pensiero divergente implica “fluidità, flessibilità, originalità” e riguarda essenzialmente la produzione di idee nuove e numerose.

\subsection*{Stimolare la creatività}
Esistono varie tecniche per stimolare la creatività. \newline
Alla base della generazione di idee vi è la  necessità di non dare nulla per scontato, nulla per vietato, nulla per immutabile. Cartesio suggeriva di non accogliere mai nulla per vero che non si conoscesse essere tale per evidenza. Analogamente, nessuna idea dovrebbe essere assurda finché questo non è dimostrato.

\subsubsection*{Tecnica dei sei cappelli}
È una tecnica ideata da Edward De Bono, attualmente considerato il massimo esponente nel campo del pensiero creativo. \newline
La tecnica si basa sulla convinzione che affrontare un problema globalmente, utilizzando tutte le capacità del nostro cervello contemporaneamente, produca scarsi risultati in termini creativi. De Bono osserva che, quando si pensa ad un determinato argomento si possono assumere sei diversi atteggiamenti mentali. Con la tecnica dei sei cappelli si deve portare un cappello alla volta, quindi si è costretti a passare attraverso tutti gli atteggiamenti separatamente, in sequenza, senza giungere affrettatamente alle conclusioni. La tecnica dei sei cappelli consente di pensare e dire cose che, altrimenti, metterebbero a rischio il nostro ego e di concentrare l’attenzione su un aspetto per volta del problema. Inoltre, i sei cappelli per pensare permettono di chiedere agli altri di assumere un certo modo di pensare come se si trattasse di una parte da recitare.
\begin{itemize}
	\item Cappello nero. Il cappello nero rappresenta la razionalità negativa: si tratta di esprimere ed evidenziare con obiettività gli elementi deboli, rischiosi o che destinano all’insuccesso un certo prodotto o progetto.
	\item Cappello giallo. Il giallo è un colore solare e positivo: rappresenta l’ottimismo, gli aspetti positivi, ossia tutte le ragioni per cui una soluzione funzionerà, fondandosi comunque sulla realtà
	\item Cappello verde. Il verde evoca l’immagine della natura, di crescita fertile: è il cappello creativo, delle nuove idee, proposte e alternative. Si tratta di ricercare altre modalità di soluzione, di produrre miglioramenti a soluzioni già individuate. De Bono suggerisce un modo per stimolare questa crescita: esporre idee provocatorie per poter uscire dagli schemi.
	\item Cappello bianco. Il bianco è un colore neutro, rappresenta i fatti e i dati oggettivi. “pensare con il cappello bianco”, afferma De Bono. “diventa una disciplina che aiuta il pensatore a separare nettamente i fatti da estrapolazioni o interpretazioni”.
	\item Cappello rosso. Il rosso suggerisce ira, rabbia ed emozioni, perciò indossare il cappello rosso significa che si devono esprimere le emozioni, i sentimenti, le sensazioni sia razionali che irrazionali senza doverne spiegare il perché.
	\item Cappello blu. Questo cappello rappresenta il controllo su tutto il processo: è con esso che si stabilisce quali altri cappelli occorre indossare e quando. Il cappello blu organizza il processo di pensiero.
\end{itemize}

\subsubsection*{Brainstorming}
Il brainstorming, letteralmente “tempesta del cervello”, è stato proposto verso il 1938 come metodo di creatività da Alex Osborn. Si tratta di una tecnica basata su una discussione di gruppo incrociata, guidata da un moderatore che mira a fare esprimere, in maniera assolutamente non univoca, il maggior numero possibile di idee su un determinato problema (Osborn, 1992). \newline
La constatazione su cui poggia questa tecnica è che le idee, una vota espresse, richiamano altre idee, e queste a loro volta, altre ancora. Un flusso continuo che si autoalimenta grazie all’apporto di tutti. Entrando in una sessione di brainstorming, ogni partecipante viene invitato ad abbandonare la logica ed i pregiudizi. Un gruppo di brainstorming si compone, generalmente, di dieci-dodici persone, opera sotto la guida di un animatore, agisce in un tempo limite di 45-50 minuti.

\section{Il piano di Marketing}
Il piano di marketing rappresenta il complesso di azioni coordinate che un’impresa realizza per raggiungere i propri obiettivi di marketing, nonché la sequenza di fasi e i tempi necessari per realizzarli. \newline
Un piano di marketing si compone delle seguenti fasi:
\begin{itemize}  
	\item Ricerca di mercato
	\item Segmentazione, targeting e posizionamento 
	\item Definizione e implementazione del marketing mix
	\item Controllo e valutazione dei risultati
\end{itemize}

\chapter{Ricerca di mercato}
Le ricerche di mercato presentano carattere esplorativo e sono effettuate sia per individuare la “rotta” aziendale, sia per accertare specifici aspetti/bisogni del mercato, come atto propedeutico alla definizione di una strategia o di un piano di fattibilità. La ricerca di mercato è, dunque, l’insieme degli strumenti utilizzati da coloro che producono beni e servizi o che promuovono idee e valori, per mantenersi in contatto con i bisogni e le necessità dei consumatori. \newline
Una ricerca di marketing si compone delle seguenti fasi:
\begin{itemize}
	\item Definizione del problema e sviluppo delle ipotesi
	\item Formulazione del disegno della ricerca
	\begin{itemize}
		\item Costruzione dello o degli strumento/i
		\item Definizione di un piano di campionamento
	\end{itemize}
	\item Racccolta dei dati
	\item Codifica e analisi dei dati
	\item Interpretazione dei risultati
	\item Presentazione finale dei risultati
\end{itemize}
La ricerca è finalizzata ad approfondire caratteristiche e le opportunità del mercato riferimento, nonché a determinare in che misura gli acquirenti differiscono tra loro in termini di bisogni, percezioni e preferenze.

\section{Individuazione della domanda}
Scelta della categoria di potenziali clienti i cui bisogni e desideri si vogliono soddisfare attraverso determinati prodotti e identificazione del numero degli acquirenti, del loro potere d’acquisto e della loro disponibilità all'acquisto. \newline
Per l’analisi delle opportunità ci si può avvalere di due tipologie di dati:
\begin{itemize}
	\item Dati primari già disponibili:
	\begin{itemize}
		\item Fonti statistiche ufficiali (Istat, Eurostat, Ministeri, enti e associazioni per la ricerca economica, associazioni di categoria, Camere di Commercio(CCIAA), banche e imprese, istituti e agenzie per il marketing, fonti informative estere)
		\item Associazioni di categoria
		\item Istituti specializzati (DOXA, Censis, Nielsen)
		\item Letteratura specialistica (riviste e data base bibliografici, atti di conferenze)
		\item Letteratura non specialistica (periodici, testate giornalistiche, pubblicazioni varie, ...)
		\item Rassegna stampa (articoli di quotidiani)
	\end{itemize}
	\item Dati secondari raccolti attraverso la messa a punto di Ricerche di Mercato rivolte ai potenziali consumatori (questionari, interviste, focus group ...)
\end{itemize}

\section{Definizione di un piano di campionamento}
La conoscenza statistica di un fenomeno può essere acquisita sia mediante una rilevazione completa delle sue manifestazioni, sia attraverso una rilevazione parziale che consenta di risalire con buona approssimazione alle caratteristiche complessive del fenomeno. La rappresentatività dei dati raccolti attraverso rilevazioni complete non è sempre superiore a quella riscontrabile utilizzando metodologie campionarie, dato che l’errore di osservazione nelle prime è spesso altrettanto consistente che nelle rilevazioni parziali.
\begin{description}
	\item[Popolazione:] qualsiasi insieme di elementi simili tra loro per una o più caratteristiche, che rappresentano l’oggetto di studio di una particolare indagine.
	\item[Campione:] si definisce quella parte limitata di popolazione che viene presa in esame.
	\item[Campionamento:] rappresenta il procedimento di individuazione del campione, cioè di un insieme di osservazioni rappresentativo della popolazione.
	\item[Parametro:] si intende la misura di una caratteristica della popolazione.
	\item[Statistica:] si intende la misura corrispondente nel campione.
\end{description}
Le indagini campionarie sono preferibili a quelle complete nei seguenti casi: 
\begin{itemize}
	\item quando la popolazione è di \textbf{dimensione} infinita o è un’entità astratta
	\item quando le rilevazioni di tutte le unità della popolazione da analizzare comportano \textbf{costi elevati.}
	\item quando i risultati della ricerca devono essere forniti in \textbf{tempi brevi.}
	\item quando la rilevazione, misurazione e controllo delle informazioni comporta la \textbf{distruzione delle unità esaminate}, come nel controllo statistico di qualità della produzione.
	\item nelle prove di mercato di un prodotto, quando esse non si possono svolgere sull'intero mercato, per motivi di \textbf{riservatezza verso la concorrenza} o per incompleta definizione di alcune variabili di marketing.
\end{itemize}

\subsubsection{Tipologie di errori di campionamento}
\paragraph{Errore casuale:} esso risulta determinato dalle fluttuazioni casuali del campione e può essere stimato ed eventualmente ridotto aumentando la numerosità del campione stesso.

\paragraph{Errore sistematico:} esso risulta più difficile da cogliere poiché direttamente connesso alle problematiche relative al piano di campionamento e quindi al metodo di rilevamento utilizzato e alle tecniche impiegate.

\subsection{Tecniche di campionamento}
Le tecniche di campionamento possono essere suddivise in due principali categorie:
\begin{itemize}
	\item quelle che danno luogo a campioni probabilistici
	\item quelle che producono campioni non probabilistici
\end{itemize}
\paragraph{Campionamento probabilistico}
Si ha un campione probabilistico quando ad ogni unità della popolazione compete una probabilità nota e non nulla di essere estratta. \newline
I tipi più comuni di campionamento probabilistico sono i seguenti:
\begin{itemize}
	\item campionamento casuale semplice
	\item campionamento stratificato
	\item campionamento per aree o per grappoli
\end{itemize}

\subsubsection{Campionamento casuale semplice}
Le unità vengono estratte singolarmente mediante un processo di scelta casuale, in modo che ogni unità abbia la medesima probabilità di venire selezionata e che la sua estrazione sia indipendente da quella delle altre. \newline
È necessario poter disporre di una lista completa e precisa delle unità di campionamento da cui estrarre i singoli elementi. Si possono adottare procedure diverse per ottenere il campione, ma esse comunque devono garantire l’equiprobabilità di estrazione ad ogni unità della lista. \newline
Un’ampia numerosità campionaria può condurre ad un campione casuale più rappresentativo.

\paragraph{Vantaggi} 
Il campionamento casuale semplice risulta il metodo migliore quando si conosce poco o nulla riguardo alla popolazione in esame. \newline
Con una numerosità elevata, il campione casuale diviene tanto più rappresentativo quanto più le caratteristiche essenziali dei dati estratti tendono a rispecchiare quelle della popolazione. \newline
La responsabilità del ricercatore è limitata, in quanto egli è tenuto a compiere soltanto alcune scelte preliminari, dal momento che la rappresentatività del campione scaturisce in maniera pressoché automatica dall'estrazione casuale.

\paragraph{Svantaggi} 
Tale campionamento richiede di poter disporre di una lista completa e corretta degli elementi che costituiscono la popolazione. \newline
Vi sono inevitabili difficoltà relativamente alla fase di contatto con gli individui estratti, il cui reperimento spesso costituisce un onere assai gravoso. \newline
Questo tipo di campionamento espone all'errore casuale, che può tuttavia essere contenuto aumentando la numerosità campionaria, determinando tuttavia un incremento dei costi complessivi della ricerca. \newline
Ritroviamo due tipologie di campionamento casuale semplice: senza e con ripetizione. \newline
Nella prima tipologia ogni unità campionaria via via estratta viene reinserita nella popolazione di cui fa parte; attraverso tale modalità le probabilità di estrazione non vengono modificate dalle successive estrazioni, dando luogo a stime più attendibili.\newline
Quando si dispone di grandi popolazioni, si assume che l’estrazione mediante la tecnica del campionamento casuale senza reinserimento non modifichi la probabilità di selezione degli altri elementi. \newline
Nelle piccole popolazioni, invece, per mantenere la casualità nell'estrazione campionaria, è preferibile porre ciascuna unità nella condizione di essere nuovamente selezionata.

\subsubsection{Campionamento stratificato}
Il campionamento stratificato viene utilizzato quando nella popolazione di riferimento sono possibili da individuare naturalmente delle sottopopolazioni o degli strati omogenei. \newline
L’universo dell'indagine viene considerato suddiviso in sub-universi esprimenti ciascuno le variabili di rilievo per l’indagine e da ogni strato si estraggono con criteri di causalità, le unità campionarie. I gruppi di unità si presentano omogenei al loro interno – rispetto ai caratteri considerati – e quanto più possibile eterogenei fra di loro.

\paragraph{Vantaggi} 
Da un campione stratificato si ottengono stime più attendibili rispetto ad un campione casuale semplice della stessa dimensione. Si ottengono, inoltre, delle stime più attendibili riducendo la numerosità campionaria. \newline
La stratificazione comporta, inoltre, una certa convenienza gestionale ed organizzativa, poiché il ricercatore può ottimizzare la realizzazione delle rilevazioni e ridurre i tempi e i costi riferendosi a variabili naturali facilmente identificabili.

\paragraph{Svantaggi}
Il campione stratificato può essere utilizzato solo nel caso in cui abbiamo informazioni sufficienti riguardo alle distribuzioni e alle proporzioni delle differenti tipologie nella popolazione che andranno a costituire i diversi strati.

\paragraph{Campionamento non probabilistico}
Nel campionamento non probabilistico le singole unità della popolazione non hanno la stessa probabilità di entrare a far parte del campione. \newline
Tra i campioni non probabilistici ritroviamo in particolare:
\begin{itemize}
	\item campionamento non probabilistico per quote
	\item campionamento non probabilistico a valanga
	\item campionamento non probabilistico accidentale
\end{itemize}

\subsubsection{Campionamento non probabilistico per quote}
\begin{enumerate}
	\item La popolazione di riferimento viene suddivisa in classi o sottogruppi omogenei secondo, a titolo esemplificativo, il genere, l’età e il titolo di studio.
	\item Dai dati dell'ultimo censimento si ricava il peso percentuale di ogni classe.
	\item Il totale delle unità campionarie viene successivamente suddiviso tra le classi in modo da rispecchiare le proporzioni esistenti nella popolazione.
	\item Si perviene, dunque, alla definizione delle quote, cioè del numero delle interviste da effettuare in ciascuna classe.
\end{enumerate}

\paragraph{Vantaggi} 
\begin{itemize}
	\item Il campionamento per quote permette di contenere i tempi e i costi della ricerca.
	\item Consente ai rilevatori di non spostarsi eccessivamente, poiché è sufficiente sostituire opportunamente un soggetto non raggiunto invece di ricontattarlo.
	\item Si tratta di una procedura particolarmente utile quando vengono richiesti risultati urgenti.
	\item Essendo questo metodo molto più flessibile di quello casuale, risulta particolarmente interessante e adeguato per chi si occupa di marketing.
\end{itemize}

\paragraph{Limiti} 
\begin{itemize}
	\item Poiché il metodo non comporta il controllo dei rifiuti a collaborare, è possibile che vi sia un minore impegno da parte dei rilevatori per ottenere l’intervista.
	\item Esiste il rischio di distorsione per quanto attiene l’inclusione di coloro che sono difficilmente reperibili.
	\item Esiste la possibilità di una sottostima della variabilità, se il rilevatore tende ad avvicinare persone tra loro simili (errore sistematico).
	\item E’ possibile una iniziale distorsione dei dati, quando le quote vengono stabilite su statistiche non aggiornate.
\end{itemize}

\subsubsection{Campionamento non probabilistico a valanga}
Il campione a valanga viene impiegato nelle indagini sulle popolazioni rare, e consiste nello scegliere un gruppo iniziale di persone, dalle quali poi ottenere nomi e indirizzi di altre unità appartenenti alla stessa popolazione. È utile ad esempio, nel caso di gruppi linguistici o etnici di piccole dimensioni o dispersi sul territorio.
\begin{description}
	\item[Prima fase] si individuano e si intervistano alcune persone in possesso delle caratteristiche che si vogliono analizzare. Questi soggetti vengono considerati informatori utili ad individuare altre persone idonee ad essere incluse nel campione.
	\item[Seconda fase] si intervistano queste persone che, a loro volta, ci permettono di contattare altri soggetti ancora da intervistare nella terza fase, e così via.
\end{description}

\subsubsection{Campionamento non probabilistico accidentale}
Il campionamento accidentale o di convenienza è la tecnica più rudimentale di selezione, in quanto il ricercatore si limita a scegliere come rispondenti le prime persone che capitano. I soggetti vengono acquisiti in modo indiscriminato fino a quando non si raggiunge l’ampiezza prestabilita dal campione. Con tale tecnica si ottiene un risparmio considerevole di tempo e di denaro, ma essa è particolarmente soggetta ad errori sistematici, quindi viene utilizzata quasi esclusivamente in indagini esplorative.

\section{Tipi di ricerca di mercato}
Le ricerche di mercato si distinguono in:
\begin{itemize}
	\item ricerche esplorative
	\item ricerche descrittive o correlazionali
	\item ricerche causali o esplicative
\end{itemize}

\subsection{Ricerche esplorative}
Sono volte a chiarire la natura di un problema e a tradurlo in specifiche ipotesi di ricerca. Come metodi si avvale di ricerca documentale e di studi e strumenti qualitativi.

\subsection{Ricerche descrittive o correlazionali}
Presuppongono una buona definizione del problema e sono volte a fornire il maggior numero di informazioni per descrivere un problema senza però stabilire rapporti di causa-effetto tra attori o di formulare previsioni. \newline
Come metodi si avvale di interviste strutturate faccia a faccia o telefoniche e di questionari.

\subsection{Ricerche causali o esplicative}
Sono volte a identificare la relazione di causa-effetto tra una o più variabili e fornire una spiegazione ai fenomeni descritti in modo da formulare previsioni per il futuro. Implicano l’esistenza di ipotesi ben definite e il ricorso a una metodologia precisa e permettono la creazione di modelli in grado di simulare strategie di marketing. Come metodi si avvale di esperimenti di laboratorio e sul campo.

\section{Tecniche di indagine}
\begin{itemize}
	\item \textbf{Qualitative:} colloqui, interviste in profondità individuali e/o collettive, focus group, tecniche proiettive (descrizione di terze persone, completamento di frasi, libere associazioni di parole, interpretazione di disegni)
	\item \textbf{Quantitative:} questionario, interviste semi-strutturate, interviste strutturate.
\end{itemize}

L’informazione qualitativa sorregge processi decisionali di due tipi: situazioni caratterizzate da un’elevata indeterminatezza e problemi ancora non chiaramente definiti, e situazioni caratterizzate da un’elevata soggettività e sottoposti a valutazioni di natura verbale. L’informazione qualitativa produce per sua natura delle indicazioni. \newline
Le indicazioni portano all'attenzione del ricercatore alcuni aspetti del problema di marketing da lui non direttamente percepiti. \newline
Riesce inoltre a cogliere la la subliminarietà: cioè la capacità dell'informazione qualitativa di andare oltre il semplice dichiarato dell'intervistato, cogliendone aspetti motivazionali, di desiderio, di aspettative. \newline
L’informazione quantitativa sorregge elementi di marketing per loro natura quantitativi e soggetti a valutazioni di natura numerica. Ciò permette la comparabilità: è possibile monitorare nel tempo un fenomeno di marketing, istituire relazioni incrociate fra misure diverse ed effettuare controlli di performance rispetto a standard di riferimento.

\paragraph{Approccio integrato}
Individuazione delle ipotesi e messa a punto dello strumento di rilevazione, avvalendosi del contributo informativo prodotto dall'indagine qualitativa; e indagine quantitativa estensiva, tesa a misurare le dimensioni del fenomeno emerse dalla fase qualitativa.

\section{Questionario}
Il questionario è uno strumento di rilevazione dei dati, ma si può anche chiamare tecnica di raccolta dei dati. \newline
Un questionario deve svilupparsi in modo logico e tale da mantenere alto l’interesse dell'intervistato. \newline
Dalla letteratura è possibile trarre alcuni suggerimenti:
\begin{itemize}
	\item raggruppare gli item che si riferiscono alla stessa area tematica ed esaurirla prima di passare alla successiva
	\item passare dalle questioni generali a quelle più specifiche
	\item porre le domande complesse in una posizione intermedia
	\item se è necessario usare una sequenza temporale, è bene porre prima  le domande riguardanti il passato, poi quelle relative al presente, quindi quelle relative al futuro
	\item affrontare prima i temi più concreti o più conosciuti, quindi quelli più astratti o meno conosciuti
	\item all’inizio del questionario dovrebbero essere poste domande a cui è facile rispondere, non dovrebbero essere percepite come minacciose né richiedere un eccessivo sforzo di memoria o di concentrazione e se possibile essere interessanti; porre alla fine del questionario le domande che possono suscitare ostilità e resistenze
	\item separare le domande che fungono da controllo dell'attendibilità delle risposte.
\end{itemize}

\subsection{Pretest}
È condotto su un numero limitato di soggetti (12-25), ma è indispensabile che questi abbiano caratteristiche simili a quelle della popolazione che si vuole studiare.
Lo scopo è miglioramento dello strumento prima che esso venga effettivamente impiegato, ciò avviene tramite l'individuazione e la correzione delle principali e debolezze tra cui:
\begin{itemize}
	\item gli errori di interpretazione
	\item le domande superflue
	\item le domande mancanti
	\item le domande inappropriate, ridondanti o confuse
\end{itemize}

\subsection{Presentazione}
La presentazione del questionario ha almeno tre funzioni:
\begin{enumerate}
	\item presentare le persone e/o l’organizzazione che la conducono
	\item indicare gli obiettivi dell’indagine
	\item sottolineare il valore della collaborazione dell'intervistato
\end{enumerate}
Le istruzioni devono comunicare con chiarezza le regole da seguire nella compilazione. Per assicurarsi una regolare compilazione è necessario riportare per iscritto anche alcuni esempi, utili soprattutto per la scelta fra le alternative di risposta. È importante la garanzia dell'anonimato, che si esprime nel cosiddetto “segreto statistico”: ciò assicura che le informazioni ricevute dagli intervistati saranno rese note solo ed esclusivamente sotto forma di dati globali.

\subsection{Titolo}
E’ necessaria molta cautela nella scelta del titolo del questionario.\newline
Un titolo può essere utile anche per incentivare la collaborazione dei soggetti. \newline
Un titolo che cita esplicitamente gli obiettivi della ricerca può essere inadeguato. Vale il suggerimento di sceglierlo dopo un’attenta rilettura del questionario ed una estrapolazione dei suoi contenuti principali.

\subsection{Linguaggio}
Deve essere uno strumento comprensibile per tutti quando deve essere somministrato ad un campione eterogeneo di soggetti.\newline
È buona regola evitare i termini tecnici o specialistici. L’uso di una terminologia inadeguata può provocare difficoltà nella comunicazione o addirittura la sua interruzione, una caduta della motivazione, imbarazzo ad ammettere di non conoscere ciò di cui si sta parlando e la tendenza a rispondere a caso pur di non chiedere un chiarimento. Anche l’uso di alcuni concetti come, ad esempio, le percentuali e le proporzioni possono risultare di difficile comprensione.\newline
È un errore usare espressioni colloquiali, dialettali o di gergo. È consigliabile usare una forma impersonale di abituale cortesia.

\subsection{Lunghezza}
Limitare il più possibile la lunghezza delle domande. Un questionario lungo consente di raccogliere maggiori informazioni e quindi di analizzare e comprendere meglio i diversi aspetti del problema oggetto di indagine. Nel contempo, però, un questionario lungo può determinare una progressiva stanchezza e riduzione della motivazione del rispondente. \newline
Una eccessiva lunghezza potrebbe provocare eventuali rifiuti a continuare o un numero più alto di risposte date a caso, a causa
di un abbassamento della motivazione del soggetto a collaborare. La somministrazione di un questionario non dovrebbe superare in ogni caso i 60 minuti. \newline
Può essere consigliabile in tal caso una pausa in corrispondenza con un cambiamento di area tematica, dopo i primi 20 minuti.

\subsection{Domande chiuse}
Domande chiuse o ad alternative fisse o strutturate: esse sono accompagnate da una lista di alternative (o categorie o modalità di risposta) fra le quali il soggetto deve scegliere quella o quelle che meglio rappresentano la sua risposta o il suo giudizio. \newline
Esse possono essere distinte in base al formato di risposta in:
\begin{itemize}
	\item domande a risposta alternativa o a scelta forzata: il soggetto può rispondere con un “sì” o con un “no”, oppure scegliendo fra due alternative opposte.
	\item domande a risposta graduata: il soggetto può scegliere tra varie alternative presentate in un certo ordine.
	\item domande con lista di preferenza o a scelta multipla: il soggetto nel rispondere deve compiere una scelta da un elenco di possibili risposte e tale scelta può cadere su una o più alternative di risposta.
\end{itemize}

\subsubsection{Vantaggi}
Sono più facili e veloci da somministrare e codificare. La lista delle categorie di risposta che accompagna una domanda chiusa può aiutare l’intervistato a comprendere meglio la domanda e a focalizzarsi proprio su quegli aspetti che interessano al ricercatore. L’uniformità della struttura di riferimento rappresentata dalle alternative di risposta facilita il confronto tra soggetti e le analisi statistiche. È più facile ottenere risposte su argomenti delicati.

\subsubsection{Svantaggi}
Le categorie di risposta predeterminate possono “suggerire” una risposta. \newline
La definizione della lista di alternative si basa su un’assunzione di coincidenza tra gli schemi del ricercatore e quelli di tutti i soggetti, tuttavia ogni intervistato può dare a una domanda un’interpretazione personale che non traspare dalle sue risposte. Alcune persone, le più interessate e motivate, possono reagire negativamente alle domande chiuse, in particolare se ritengono che nessuna delle categorieproposte può esprimere l’opinione che hanno elaborato.

\subsection{Domande aperte}
Domande aperte: le possibilità di risposta non sono stabilite a priori e non suggerendo alcuna alternativa di risposta consentono al soggetto di rispondere in base alla propria struttura di riferimento e di rilevare ciò che è più rilevante per il soggetto stesso.
\subsubsection{Vantaggi}
Permettono ai soggetti di rispondere in base alla propria struttura di riferimento senza essere influenzati dalle alternative di risposta proposte dal ricercatore. \newline
Sono utili quando non è possibile prevedere tutta la lista delle categorie di risposta o quando questa è troppo numerosa. \newline
Secondo alcuni autori, sono meno sensibili agli effetti della desiderabilità sociale delle risposte.
\subsubsection{Svantaggi}
Gli intervistati sono spesso più restii ad esprimere la loro opinione. Sono più difficili da codificare e analizzare. Richiedono un compito di recupero delle informazioni più gravoso cognitivamente e potenzialmente soggetto a varie distorsioni.

\subsection{Scale di giudizio}
Esse cercano di rilevare gli atteggiamenti e le valutazioni del soggetto chiedendogli di esprimerli nei termini di un giudizio categoriale o numerico. \newline
Si distinguono in:
\begin{itemize}
	\item scale tipo Likert
	\item scale a somma 100
	\item scale a 100 punti
	\item differenziale semantico
\end{itemize}

\subsubsection{Scale tipo Likert}
Esse sono costituite da categorie di risposta ordinate, di numeri, di linee o di categorie verbali. Il cui numero è variabile, anche se generalmente sono 5 o 7. Ai soggetti si chiede di scegliere una sola alternativa. Le espressioni verbali possono essere le seguenti: assolutamente d’accordo, abbastanza d’accordo, poco d’accordo, né d’accordo né in disaccordo, poco in disaccordo, abbastanza in disaccordo, assolutamente in disaccordo.

\subsubsection{Scale a somma 100}
Esse prevedono che il soggetto giudichi le alternative proposte avvalendosi di un criterio di valutazione basato sulla distribuzione di un totale di 100 punti tra le alternative stesse. Il soggetto deve indicare le percentuali relative a ciascuna alternativa in modo che la somma delle stesse risulti pari a 100. Alla base della procedura si colloca l’assunto che il giudizio da parte dei soggetti avvenga previa considerazione d’insieme di tutte le alternative di risposta.\newline
Il totale dei punti attribuiti ad ogni alternativa è infatti il risultato di molteplici confronti messi in atto al fine di stabilire la relazione esistente fra ciascuna alternativa e tutte le altre.\newline
Uno dei limiti applicativi di questo tipo di scala si evidenzia nel caso di un numero piuttosto elevato di alternative di risposta da giudicare. È infatti ipotizzabile la difficoltà incontrata dal soggetto nel fornire una valutazione basata sul confronto contemporaneo di tutte le alternative. Per ovviare a tale limite si possono raggruppare più alternative in un singolo item.

\subsubsection{Scala a 100 punti}
Esse offrono la possibilità al soggetto di attribuire ad ogni alternativa di risposta 100 punti. Tale procedura permette di assegnare un valore numerico da 1 a 100 ad ogni categoria di risposta, sostituendo così il metodo tradizionale che prevede una classificazione delle alternative per ordine di importanza. Il vantaggio di questa scala deriva dal fatto che il soggetto non viene obbligato a fare una scelta tra le alternative di risposta ma è libero di attribuire a ciascuna di esse il valore che ritiene più opportuno.

\subsubsection{Differenziale semantico}
Il soggetto deve valutare un concetto scegliendo uno dei sette punti interposti tra due polarità, definite da aggettivi di significato opposto. Il soggetto ha il compito di comunicare le sue prime impressioni senza “pensare troppo a qual è la risposta giusta: lo scopo è quello di scoprire quali sono i sentimenti che le persone provano nei confronti di certi argomenti, favorendo la formulazione di una risposta sincera.

\subsection{Domande filtro}
Esse mirano a rilevare se a un soggetto vanno rivolte le domande successive del questionario. Prima di affrontare un argomento con l’intervistato può essere utile infatti sapere se egli possiede su questo un certo grado di competenza o di esperienza. Una domanda condizionata segue una domanda filtro. Operando una selezione dei soggetti, consentono un notevole risparmio di tempo, tuttavia si avrà una diminuzione della numerosità del campione, del quale bisognerà controllare di volta in volta la rappresentatività.

\subsection{Mettere a proprio agio l'intervistato}
\subsubsection{domande imbarazzanti/intrusive/delicate}
È importante non mettere in imbarazzo l’intervistato, soprattutto quando si affrontano argomenti delicati oppure quando le persone possono non conoscere bene l’argomento.

\textit{Formulazione sbagliata:} Hai mai fatto uso di droghe?
\textit{Formulazione corretta:} Molte persone hanno fatto uso droghe in alcuni momenti della loro vita, a lei è successo qualche volta? \newline
Quando si affrontano temi delicati è importante presentare l'argomento come una situazione già affrontata da altri su cui non si può rispondere senza riflettere; una risposta eventualmente positiva non farebbe sentire a disagio chi risponde perché gli è appena stato ricordato, nella domanda stessa, che gli altri si sono trovati nella sua stessa situazione.

\subsubsection{Domande di argomento sconosciuto}

\textit{Formulazione sbagliata:} Secondo voi quale corrente artistica sta maggiormente influenzando i giovani musicisti italiani? 

\textit{Formulazione corretta:} Avete qualche idea su quale corrente artistica stia maggiormente influenzando i giovani musicisti italiani? \newline
Formulare quindi la domanda in modo da mettere a proprio agio l'intervistato anche nel caso di una risposta negativa che denota ignoranza sull'argomento.

\subsection{Domande da evitare}
\subsubsection{Domande con negazione (peggio se multipla)}
Per quanto possibile, vanno evitate formulazioni di domande o affermazioni in cui si fa uso della negazione o della doppia negazione.
\textit{Formulazione sbagliata:} Penso che gli immigrati extracomunitari NON meritino la fiducia degli italiani.
\textit{Formulazione corretta:}
Penso che gli immigrati extracomunitari meritino la fiducia degli italiani. \newline
Nel dare la risposta un intervistato farà prevalere la reazione al termine, immigrati, invece di attenersi al significato linguistico della frase, se xenofoba potrà rispondere "completamente contrario", sottintendendo "agli immigrati", equivocando sul significato della domanda.

\subsubsection{Domande doppie}
Sono doppie tutte le domande in cui una parte della risposta potrebbe positiva e l’altra negativa.
\paragraph{Formulazione sbagliata:} Si ritiene soddisfatto delle mansioni svolte e della posizione occupata nel suo attuale lavoro?
\paragraph{Formulazione corretta:} Si ritiene soddisfatto delle mansioni svolte nel suo attuale lavoro? E della posizione occupata?

\subsubsection{Domande ambigue}
È importante che le domande siano formulate in modo da contenere informazioni sufficienti a non risultare ambigue. Se si vuole che i gli intervistati rispondano tutti alla medesima domanda bisogna evitare che gli intervistatori siano costretti ad aggiungere “parole proprie” per specificare una domanda incompleta.
\paragraph{Formulazione sbagliata:} La mattina consuma una colazione? \newline
\begin{itemize}
	\item Non chiarisce da cosa sia costituita una colazione
	\item Non è chiaro fino a che ora del mattino un pasto possa essere considerato una colazione
	\item Non è chiaro se la domanda si riferisce ad un consumo abituale o a un giorno preciso
\end{itemize}

\paragraph{Formulazione corretta:} Per i nostri scopi consideri colazione un pasto costituito almeno da una bevanda (te, latte, caffè, ...) e un alimento come brioches, cereali, biscotti, toast o frutta, consumato prima delle 10 del mattino. Secondo questa definizione negli scorsi 7 giorni quante volte ha consumato una colazione?

\paragraph{Formulazione sbagliata:} Quanto è importante che un telefonino abbia la fotocamera?

\paragraph{Formulazione corretta:} Quanto è importante \textit{per te} che un telefonino abbia la fotocamera?
Nella prima formulazione della domanda dell'esempio non c'è una chiara indicazione del criterio da considerare e alcuni intervistati potranno dare una risposta basandosi sulla loro opinione, mentre altri valuteranno le caratteristiche di un telefonino facendo riferimento genericamente ai possibili acquirenti.

\subsubsection{Domande pilotanti}
E’ il caso in cui il testo della domanda indirizza l’intervistato, più o meno apertamente, verso una alternativa di risposta.
\paragraph{Formulazione sbagliata:} Quanto è \textit{positivo} il suo giudizio su Matteo Renzi?

\paragraph{Formulazione corretta:} Qual è il suo giudizio su Matteo Renzi?

\subsubsection{Domande oggettive} 
Bisogna preferire domande oggettive
\paragraph{Domanda non oggettiva:} Studi il pomeriggio?
\paragraph{Domanda oggettiva:} Quante ore hai studiato oggi pomeriggio?
\paragraph{Domanda non oggettiva:} Ti piace andare a teatro?
\paragraph{Domanda oggettiva:} Quante volte sei andato a teatro nell'ultimo mese?

\chapter{Segmentazione, targeting e posizionamento}
\section{Segmentazione}
La segmentazione è un processo mediante il quale il mercato viene suddiviso in un numero limitato di segmenti, ovvero di gruppi di consumatori, sufficientemente omogenei come motivazioni e comportamenti al loro interno e, invece, sufficientemente eterogenei fra di loro, così da aspirare a diversi marketing mix.(Collesei, 1994). \newline
La segmentazione richiede una dose di creatività, e la base scelta per segmentare il mercato è altrettanto importante quanto il maggiore o minore grado di che si ritiene di accettare all’interno dei segmenti (Abell \& Hammond, 1986). \newline
Nella definizione della propria strategia di segmentazione le imprese debbono valutare attentamente l’attrattiva dei segmenti e decidere in quale segmento posizionarsi. Nel fare ciò, devono fare molta attenzione a non essere attratte dalla cosiddetta miopia di marketing . I segmenti di maggiore consistenza, a motivo delle maggiori vendite potenziali che sembrano offrire, sono quelli che attraggono di più l’interesse delle imprese, specie di quelle di grandi dimensioni. Spesso tuttavia in questi segmenti si verifica un’elevata concentrazione di concorrenti. Risulta perciò più difficile per l’impresa ottenere una adeguata quota di mercato. Al contrario, la scarsa presenza di concorrenti, in particolare di dimensioni elevate, rende più congeniale alle piccole imprese la scelta di segmenti minori. Tale strategie viene definita di nicchia. \newline
Una volta effettuata la segmentazione, e quindi individuata una suddivisione del mercato in segmenti, si presentano all’impresa due principali opportunità:
\begin{itemize}
	\item \textbf{segmentazione concentrata}: scegliere un unico segmento in funzione del quale creare un unico prodotto;
	\item \textbf{segmentazione multipla}: scegliere più segmenti per ciascuno dei quali proporre differenti prodotti.
\end{itemize}

\subsection{Vantaggi della segmentazione}
\begin{itemize}
	\item Consente di individuare e confrontare con maggiore chiarezza le opportunità di mercato.
	\item Migliora la conoscenza sul comportamento del compratore.
	\item Permette di studiare i prodotti e servizi in base alle specifiche esigenze dei compratori.
	\item Promuove la sensibilità dell’impresa a percepire i mutamenti che si verificano nella domanda.
	\item Facilita l’analisi del mercato e della concorrenza, nonché la scelta del posizionamento.
	\item Permette di aumentare la fedeltà dei clienti alla marca.
\end{itemize}

\subsection{Svantaggi della segmentazione}
\begin{itemize}
	\item Per ciascun segmento di mercato è necessario sostenere distinti costi di ricerca, sviluppo e lancio di prodotti. Il costo totale che ne risulta è ovviamente superiore a quello che si avrebbe se il prodotto fosse uno solo.
	\item In particolare l’impresa può sostenere costi più elevati di ricerche di mercato poiché le conoscenze del comportamento dei potenziali compratori debbono essere approfondite e ciò per più segmenti.
\end{itemize}

\section{Posizionamento}
Il posizionamento può essere definito come l’insieme di iniziative volte a definire le caratteristiche del prodotto e ad impostare il marketing mix più adatto per attribuire una certa posizione al prodotto nella mente del consumatore (Kotler, 1984). \newline
Il posizionamento è una tecnica di marketing finalizzata ad associare un’idea/valore positiva/o ad un prodotto, nella mente del consumatore. \newline
{\Large Esempi:}
\begin{itemize}
	\item Mercedes: stile ed eleganza
	\item BMW: prestazioni elevate
	\item Volvo: sicurezza
	\item Fiat: qualità/prezzo
\end{itemize}
	
\subsection{Processo di posizionamento}
\begin{itemize}
	\item Individuazione delle richieste del consumatore.
	\item Individuazione dei concorrenti.
	\item Determinazione delle posizioni dei concorrenti.
	\item Scelta della posizione.
	\item Verifica degli effetti della scelta:
	\begin{itemize}
		\item In funzione delle preferenze dei clienti
		\item In relazione all’ampiezza della popolazione obiettivo
		\item In riferimento alle risorse dell’impresa
		\item In relazione agli intermediari
	\end{itemize}
	\item Realizzazione e controllo della posizione.
\end{itemize}

\subsection{Strategie di posizionamento}
\begin{itemize}
	\item \textbf{Posizionamento di attributo.} L’impresa si posiziona su un determinato attributo o caratteristica, senza esplicitamente affermare nessun beneficio (es. l’università più antica; albergo più panoramico).
	\item \textbf{Posizionamento di vantaggio o beneficio.} Il prodotto promette un vantaggio (Volvo - più sicura; AZ Complete White -denti più bianchi).
	\item \textbf{Posizionamento di impiego/applicazione.} Il prodotto viene presentato come il migliore in un certo campo di applicazione (es. Nike per le scarpe da corsa; Google per i motori di ricerca).
	\item \textbf{Posizionamento di categoria merceologica.} L’impresa punta ad essere il leader di un determinato settore merceologico (Post-it nel settore dei foglietti colorati semi adesivi).
	\item \textbf{Posizionamento competitivo.} Viene affermato che il prodotto è superiore o differente rispetto ad un analogo prodotto della concorrenza.
\end{itemize}

\subsection{Il posizionamento del valore}
È necessario inoltre che l’impresa definisca la posizione dei propri prodotti in riferimento ai relativi prezzi, in funzione del fatto che i clienti associano ad un determinato prezzo un certo valore del prodotto. \newline
Kotler propone cinque strategie:
\begin{enumerate}
	\item di più a più (prodotti ad alta qualità e prezzo elevato es. Chanel, Ferrari, ...)
	\item di più per lo stesso (prodotti ad alta qualità e prezzo medio es. Lexus)
	\item lo stesso a meno (prodotti a media qualità e prezzo basso, Acqua Guizza, alcuni prodotto di marca all’interno dei Discount)
	\item meno per molto meno (prodotti a bassa «qualità» e prezzo basso, es. Raynair, “prodotti a un Euro”)
	\item di più a meno (prodotti ad alta qualità e prezzo
	basso, es. negozi outlet)
\end{enumerate}

\subsection{Riposizionamento}
Il riposizionamento o rebranding di un marchio è quel processo attraverso il quale un prodotto, anche attraverso prodotti correlati, viene reimmesso nel mercato con una diversa identità. \newline
Riposizionare l’immagine di un’azienda e dei suoi prodotti è un’operazione di comunicazione e marketing molto delicata e complessa. Questo tipo di operazione si porta a termine per vari motivi tra cui: acquisire nuove fasce di mercato, adeguamento all'evoluzione del mercato, far fronte a un nuovo concorrente, per favorire un ciclo di vita più lungo del prodotto/servizio ecc. \newline
Si può parlare di:
\begin{itemize}
\item \textbf{rebranding totale} : si verifica un cambiamento radicale del logo, dell'immagine, delle strategie di marketing e di vendita, delle politiche pubblicitarie;
\item \textbf{rebranding parziale} : si verificano piccole modifiche volte a migliorare la percezione del marchio da parte dei consumatori.
\end{itemize}

\subsubsection{Co-branding}
American Marketing Association (AMA) definisce il brand come “un nome, un termine, un segno, un simbolo, un disegno o una loro combinazione che identifica un prodotto o servizio di un venditore e che lo differenzia da quello del concorrente”. Il co-branding può essere definito come una forma di cooperazione tra due marchi, avente il fine di trasferire specifiche valenze positive da una marca all’altra – dalla marca invitata che gode una specifica notorietà, alla marca ospitante – ed ottenere un prodotto di maggiore qualità. Si ha co-branding (condivisione di marchi) quando un prodotto è contrassegnato da marchi che fanno riferimento a proprietari differenti. \newline
Tipologie:
\begin{itemize}
	\item \textbf{Funzionale:} la marca ospitante si accorda con un partner avente specifiche caratteristiche distintive/elementi tangibili; il consumatore percepirà il prodotto caratterizzato da una qualità superiore rispetto alla restante offerta del mercato, in quanto le qualità positive del secondo brand si trasferiscono sull’altro e viceversa.
	\item \textbf{Simbolico:} consiste nell’associare alla marca del produttore una seconda marca, caratterizzata da attributi simbolici di tipo psico-sociale o esperienziali addizionali. Consente, per una durata generalmente breve, di mirare a quel segmento di clientela potenziale che presenta una consonanza particolare con la marca invitata.
\end{itemize}
Vantaggi:
\begin{itemize}
	\item Incremento del valore percepito dell’offerta, con conseguente possibilità di	aumento del prezzo. L’assemblaggio di una tecnologia nota consente un aumento notevole del prezzo di un prodotto di marchio non noto (Es. Fiat e Chrysler).
	\item Arricchimento della gamma, l’arricchimento della gamma delle soluzioni offerte al consumatore consente di aumentare anche la soddisfazione complessiva (Es. Fiat e Gucci).
	\item Incrementare le vendite, spesso acquisendo quote di clienti del marchio partner. Coca Cola grazie all’accordo con Nutrasweet, azienda leader nel settore di prodotti dietetici, ha introdotto la Diet Coke.
	\item Capillarità della distribuzione dei prodotti offerti: sono frequenti accordi di marketing tra imprese che, pur non operando negli stessi settori, dispongono di risorse complementari. (Es. Mc Donald’s e Agip).
	\item Economie di costo: attraverso la cooperazione nelle diverse aree operative e strategiche del marketing, le imprese possono ridurre i costi legati ad iniziative promozionali, alla distribuzione dei prodotti, all’accesso ai media, all’acquisizione di competenze necessarie alla commercializzazione di un nuovo prodotto.
	\item Accrescere le risorse immateriali: qualsiasi collaborazione è in grado di arricchire il patrimonio immateriale delle imprese partner, attraverso la condivisione di esperienze, competenze e disponibilità. Di conseguenza, può permettere alle singole imprese partner di guadagnare posizioni nella continua ricerca del vantaggio competitivo.
\end{itemize}
\chapter{Marketing mix}
Il marketing mix è la combinazione delle variabili controllabili di marketing che l’impresa impiega al fine di conseguire gli obiettivi stabiliti nell’ambito del mercato di riferimento. Definire un marketing mix significa sviluppare una proposta di valore complessiva attraverso la combinazione e l’integrazione di alcuni fattori critici. \newline
Le 4 macrovariabili che costituiscono il marketing mix sono:
\begin{enumerate}
	\item prodotto (product)
	\item prezzo (price)
	\item punto vendita e distribuzione (place)
	\item promozione (promotion)
\end{enumerate}

Alle 4 P del marketing mix si tende oggi ad  aggiungerne una quinta, di fondamentale rilevanza: il personale. Il fattore umano è diventato determinante per le attività di marketing, soprattutto nell'ambito dei servizi. 

\section{Prodotto}
Si definisce prodotto qualsiasi bene o servizio scambiabile sul mercato che può rispondere alle esigenze di un compratore. È tutto quello che il compratore considera quando decide per l’acquisto. \newline
Un servizio è qualsiasi attività o vantaggio che una parte può scambiare con un’altra, la cui natura sia essenzialmente intangibile e non implichi proprietà di alcunché. \newline
Le caratteristiche distintive del servizio sono:
\begin{itemize}
	\item l’intangibilità
	\item la contemporaneità di erogazione e fruizione
	\item il coinvolgimento diretto del fruitore nella sua realizzazione
	\item la variabilità, dovuta alla componente personale
	\item la deperibilità, poiché i servizi non si possono restituire/rivendere
\end{itemize}
L’idea che sta alla base del concetto di prodotto è che i consumatori non acquistano solo aspetti fisici, ma ricercano soprattutto un mezzo per soddisfare le loro esigenze e i loro desideri. In sostanza i consumatori ricercano benefici: le persone non comprano le cose solo per quello che servono, ma anche per quello che significano. \newline
Compito dell’operatore di marketing è di scoprire i benefici attesi dal consumatore e di offrirglieli, e non solo offrire attributi. \newline
Classificazione dei prodotti:
\begin{itemize}
	\item \textbf{Beni destinati alla produzione}: prodotti/servizi acquistati per essere impiegati nella produzione di altri beni/servizi.
	\item \textbf{Beni di consumo}: utilizzati dai consumatori/fruitori finali.
	\begin{itemize}
		\item Beni durevoli : producono la loro utilità nel tempo.
		\item Beni non durevoli : esauriscono la loro utilità in un’unica
		soluzione o in pochi usi ripetuti (alcool test, penne, ...).
		\item Beni di largo consumo ad acquisto frequente ( prodotti alimentari, sigarette ).
		\item Beni di largo consumo con forte preferenza di marca
		( abbigliamento, arredamento ).
		\item Beni ad acquisto saltuario e ponderato (abitazioni).
		\item Beni speciali (beni di lusso)
	\end{itemize}
    \item \textbf{Servizi}: prestazioni derivanti da una attività (di persone o di organizzazioni) o dalla disponibilità temporanea di un prodotto.
\end{itemize}
Attributi del prodotto:
\begin{itemize}
	\item forma, colore, design;
	\item qualità;
	\item dimensione, peso, ingombro;
	\item packaging;
	\item marca;
	\item servizi ai clienti;
	\item garanzia.
\end{itemize}
La forma di un prodotto e più in generale il suo design – ossia lo stile che caratterizza gli oggetti e che riguarda tanto gli aspetti estetici , quanto il colore e i materiali – rappresenta uno degli elementi più significativi che contraddistinguono non sono i prodotti di consumo, ma anche i beni destinati alla produzione. \newline
Il design consente di differenziare i prodotti ed è quindi legato anche al comportamento psicologico del compratore. \newline
Il design non è soltanto un attributo di carattere estetico; spesso le imprese studiano in modo approfondito il design per migliorare l’efficienza nella produzione, per rendere più facile l’utilizzazione del prodotto da parte del consumatore e per rendere il prodotto più sicuro. \newline
L’importanza del design non si limita al prodotto in sé, ma riguarda anche la sua presentazione cioè l’esposizione della merce nel negozio, attraverso l’uso del “display”("lineare espositivo"  in italiano, cioè il supporto su cui viene esposto il prodotto). \newline
Di estrema importanza nel design dei prodotti è anche l’uso del colore. I colori caldi favoriscono dunque l’acquisto d’impulso, mentre i colori freddi (blu e verde) sono più rilassanti e più adatti all’acquisto di beni durevoli. 

\subsection{Marca}
Per marca intendiamo un nome, un termine, un simbolo, un design o una combinazione di questi, che mira ad identificare i beni o servizi di un’impresa e a differenziarli da quelli di altre imprese.
La marca consente di esprimere le prestazioni tecniche di un prodotto, la forza dell’azienda, il modo con cui l’impresa fa impresa, la sua essenza profonda e i suoi valori. \newline
Un elemento fondamentale della marca è la “reputazione dell’azienda”. La comunicazione deve spingersi a comunicare ciò che c’è dietro “le quinte” del prodotto. \newline
La marca deve comunicare la stabilità e la sicurezza del prodotto e dell’azienda. \newline
La storia della marca contribuisce ad aumentare il valore. La marca e il suo vissuto rassicurano il cliente sulla validità della scelta. \newline
Principali funzioni della marca:
\begin{itemize}
	\item comunicazione e informazione, non solo nei confronti dei clienti ma di tutti gli interlocutori dell’impresa, esterni e interni;
	\item conferimento di un posizionamento al prodotto;
	\item differenziazione;
	\item conferimento di identità all’impresa;
	\item evocazione di immagini e percezioni;
	\item promessa di valore;
	\item garanzia di qualità.
\end{itemize}
Quando parliamo di marca ci possiamo riferire a varie tipologie della stessa. Un’importante distinzione è quella tra marca del produttore e marca del venditore. La prima è la marca apposta dal produttore del prodotto (es. Cirio) e viene denominata marca industriale. La seconda è quella apposta dal venditore del prodotto: in genere, si tratta di una catena di supermercati (es. Coop) o di grandi magazzini che pongono in vendita dei prodotti realizzati da altre imprese con il proprio nome.

\subsection{Packaging}
La confezione da semplice strumento di imballo del prodotto utile a fini di conservazione e di movimentazione è divenuta, per molti prodotti, un importante strumento di marketing, sia per il consumatore per il quale svolge importanti funzioni di individuazione del prodotto, sia per i distributori per i quali svolge funzioni espositive e promozionali.
\begin{itemize}
	\item \textbf{Imballo primario}: rappresenta l’essenziale contenitore che trattiene il prodotto e che ne consente la conservazione.
	\item \textbf{Imballo secondario}: si riferisce a contenitori addizionali che possono essere aggiunti a fini protettivi o per esigenze di marketing.
	\item \textbf{Confezione display}: costituita dal packaging necessario per esporre il prodotto sul punto vendita.
	\item \textbf{Confezione di stoccaggio (terzario)}: necessaria per trasportare la merce o immagazzinarla nei magazzini.
\end{itemize}

L'imballaggio o "packaging" dei prodotti rappresenta una delle principali cause della produzione dei rifiuti, con conseguente problema di smaltimento.

\subsection{Servizi ai clienti}
I confini tra prodotti e servizi tendono a diventare sempre più sfumati e dinamici.
I classici servizi di assistenza ai clienti pre e post vendita sono:
\begin{itemize}
	\item servizio informazioni;
	\item servizio credito;
	\item servizio manutenzione;
	\item servizio reclami.
\end{itemize}

La tendenza attuale è verso una sempre maggiore personalizzazione dei servizi, grazie alla gestione di un data-base contenente informazioni sui singoli clienti.

\subsection{Garanzia e certificazioni}
La garanzia rappresenta un altro importante attributo e strumento di differenziazione del
prodotto dalla concorrenza. Ci sono due tipi di garanzia:
\begin{itemize}
	\item \textbf{Garanzia implicita:} la capacità del prodotto di svolgere la funzione promessa e di non recare rischi per la sicurezza della persona.
	\item \textbf{Garanzia esplicita e certificazioni di qualità:} comunicano al cliente un determinato livello qualitativo del prodotto e gli forniscono un servizio di assistenza valido entro un certo arco temporale.
\end{itemize}

\subsection{Qualità del prodotto}
Il cliente valuta la qualità del prodotto:
\begin{itemize}
	\item \textbf{all’atto della scelta:} confrontando i benefici attesi e quelli percepiti dagli attributi tangibili ed intangibili.
	\item \textbf{nell’uso/fruizione del prodotto/servizio:} questo comporta la conferma o la modifica del giudizio di	qualità precedentemente espresso.
\end{itemize}
Fattori su cui il cliente fonda il proprio giudizio:
\begin{itemize}
	\item fisico-strutturali (packaging, design, tecnologia)
	\item funzionali (prestazioni, sicurezza, facilità d’uso)
	\item di servizio (pre/post vendita, accessibilità, reperibilità)
	\item psicologici (estetica, design, moda, comunicazione)
	\item di immagine (del prodotto e della marca)
\end{itemize}

\subsubsection{Soddisfazione del cliente}
Il \textit{modello Servqual} di Parasuraman, Zeithaml e Berry si propone di fornire una misura della qualità percepita di un servizio (customer satisfaction), e quindi della
soddisfazione del cliente, attraverso un confronto tra:
\begin{itemize}
	\item le aspettative/attese con cui il cliente si accosta alla tipologia di
	servizio
	\item le percezioni del servizio avvenute dopo il consumo/utilizzo
\end{itemize}

Questo confronto è operato attraverso una metodologia detta paradigma della discrepanza. Essa si basa su un criterio sottrattivo tra livello delle percezioni di un prodotto/servizio con il livello di aspettative in relazione a quella tipologia di prodotto/servizio. La soddisfazione è intesa come stato psicologico derivante da un gap tra la valutazione dell'avvenuta esperienza di consumo e le attese del consumatore in merito a tale esperienza.

\subsection{Ciclo di vita}
Il ciclo di vita descrive l’andamento delle vendite di un prodotto nel tempo attraverso un modello ideale: una curva crescente e poi decrescente.

Tipologie di ciclo di vita:
\begin{itemize}
	\item prodotto a vita lunga (introduzione, crescita, maturità)
	\item prodotto di moda (introduzione e crescita)
	\item prodotto gadget (introduzione)
\end{itemize}

\subsubsection{Introduzione}
Inizia quando il prodotto viene introdotto nel sistema produttivo. In questa fase viene stimolata la “domanda primaria” che mira a convincere i consumatori circa i benefici che possono ottenere dall’acquisto del nuovo prodotto. I prezzi in genere sono alti, ma lo sono anche i costi di distribuzione e di promozione. Le vendite crescono molto lentamente e i profitti sono modesti.
Lo scopo dell’impresa è creare la domanda globale il più rapidamente possibile per uscire dalla fase di incertezza.
 
Lo scopo viene raggiunto tramite il conseguimento di questi obiettivi:
\begin{itemize}
	\item Rendere nota l’esistenza del prodotto
	\item Informare il mercato sui vantaggi dell’innovazione
	\item Incoraggiare gli acquirenti a provare il nuovo prodotto
	\item Introdurre il prodotto nella rete di distribuzione
\end{itemize}

\subsubsection{Sviluppo}
Se i primi consumatori, dopo aver adottato il prodotto ripetono i loro acquisti perché sono soddisfatti, le vendite cominciano a crescere . I prezzi restano alti poiché la domanda
è forte; i costi di produzione e di distribuzione sono ora ripartiti su volumi elevati di vendita determinando una crescita dei profitti. La domanda in espansione facilita l’entrata nel mercato di nuove imprese aumentando la concorrenza. 
Lo scopo dell’impresa è ampliare ed estendere il mercato, dato che la domanda è espandibile.

Lo scopo viene raggiunto tramite il conseguimento di questi obiettivi:
\begin{itemize}
	\item Adottare un sistema di distribuzione intensivo
	\item Migliorare il prodotto, aggiungendo nuove caratteristiche
	\item Rafforzare il sistema di comunicazione basato sul posizionamento prescelto
\end{itemize}

\subsubsection{Maturità}
La domanda rallenta ulteriormente. Il numero dei concorrenti tende ad essere molto elevato, determinando l’uscita dal mercato delle imprese più deboli. I profitti cominciano a
calare poiché i prezzi sono spinti verso il basso dalla competizione. In questa fase le imprese aumentano le spese di ricerca e sviluppo allo scopo di migliorare i prodotti, nonché quelle di promozione e di pubblicità.
Lo scopo dell’impresa è mantenere e ampliare la quota di mercato conquistando un vantaggio competitivo difendibile sui concorrenti.

Lo scopo viene raggiunto tramite il conseguimento di questi obiettivi:
\begin{itemize}
	\item Migliorare le qualità del prodotto e allargare le sue prestazioni
	\item Esplorare nuove nicchie
	\item Adottare un marketing relazionale one to one che pone l’accento sulla soddisfazione a lungo termine della clientela allo scopo di creare e mantenere la fedeltà dei clienti esistenti
\end{itemize}

\subsubsection{Declino}
Le vendite cominciano a diminuire in seguito al cambiamento dei gusti del consumatore, all’introduzione di nuovi prodotti sostitutivi e all’innovazione tecnologica. In questa fase generalmente l’organizzazione non investe più, poiché i costi sono superiori ai ricavi.

\section{Prezzo}
\subsection{Per il consumatore}
Al prezzo corrisponde il valore percepito del prodotto, che è
in relazione:
\begin{itemize}
	\item alle caratteristiche funzionali (qualità, prestazioni)
	\item alle valenze psicologiche (immagine associata al prodotto e alla marca)
	\item ai servizi (garanzia, assistenza pre e post vendita, condizioni di pagamento)
\end{itemize}
Perché il consumatore acquisti un dato prodotto è necessario che il prezzo ricada entro un certo intervallo, che può variare nel tempo, per le diverse circostanze e sulla base degli stimoli che riceve dall’impresa.
Se il prezzo è troppo basso il consumatore diventa sospettoso: potrebbe essere merce falsa o rubata, se è troppo alto decide che ìl prezzo è superiore al valore che attribuisce al prodotto.

\subsection{Per l'impresa}
Il prezzo è una variabile di marketing mix.
Per l’impresa esistono almeno tre criteri per determinare il
prezzo:
\begin{enumerate}
	\item il costo di produzione
	\item l’andamento della domanda
	\item il comportamento della concorrenza
	%\item canale di distribuzione
\end{enumerate}
Il prezzo deve permettere all'impresa il recupero dei costi
sostenuti e il conseguimento dell'utile.
I ricavi e i costi hanno andamenti diversi. Esiste un punto di equilibrio chiamato “break-even point” dove costi e ricavi sono uguali. Questo punto è determinato dal numero di prodotti venduti: al di sotto di tale numero si producono perdite, al di sopra di realizza utile.

\paragraph{1)Strategia di prezzo basata sui costi}
Il costo rappresenta un livello al di sotto del quale l’impresa non può scendere per un periodo di tempo lungo senza compromettere non soltanto la redditività, ma anche la sopravvivenza.
Nel definire i prezzi molte imprese aggiungono un “ricarico” alle stime di costo. Tale metodo (cost plus pricing) consiste nel determinare il prezzo sommando al costo una quota di profitto sperato (es. ai costi aggiunto il 10\%).

\paragraph{2)Strategia di prezzo basata sulla domanda}
Determinazione del prezzo sulla base del valore. Viene stimato il prezzo massimo che il consumatore è disposto a pagare per il prodotto. Viene fissato un prezzo inferiore, in modo da lasciare un “surplus al consumatore”.
Le situazioni che si possono presentare sono molte, ma si
possono ridurre a due:
\begin{itemize}
	\item la domanda è bassa, ma si presume che possa successivamente aumentare quando il consumatore abbia avuto modo di conoscere il prodotto.
	\item la domanda è elevata, ma si presume che debba diminuire o per saturazione del mercato o perché la concorrenza riuscirà a sottrarre quote di mercato.
\end{itemize}

\paragraph{3)Strategia di prezzo basata sulla concorrenza}
Se un’impresa è in posizione di leadership per dimensione,
caratteristiche del prodotto, può entro certi limiti, ignorare il comportamento dei concorrenti e decidere di cambiare i prezzi (i prezzi degli pneumatici sono fissati dalla Michelin).
Se invece i concorrenti sono agguerriti e l’impresa è di piccole
dimensioni, il livello del proprio prezzo rispetto a quello della concorrenza va attentamente valutato (al livello di mercato o al di sotto dei livelli di mercato).

\subsubsection{Le politiche di prezzo} In generale esse variano nelle fasi di vita del prodotto: prezzi alti nella fase di introduzione, prezzi bassi per consentire l’estensione nelle fasi successive. 
Differenti segmenti di mercato rispondono a differenti prezzi di linee di prodotto.
Il servizio ai clienti è correlato al prezzo; prezzi bassi in genere sono associati a un modesto servizio al cliente.

\section{Punto vendita e distribuzione}
Per sistema di distribuzione fisica o logistica si intende il coordinamento delle attività dell’impresa che mirano a trasferire materialmente i prodotti al consumatore finale nei tempi attesi.
\paragraph{Politica distributiva} Le decisioni e le azioni dell’impresa relative ai canali di distribuzione, alle scelte e alla valutazione degli intermediari commerciali, all’organizzazione e alla gestione della forza di vendita.

\paragraph{Canale di distribuzione}
L’insieme di persone e di istituzioni che svolgono le funzioni necessarie per trasferire il prodotto
dal produttore al cliente.

\subsection{Classificazione dei canali distributivi}
\paragraph{Canale diretto}
Il produttore vende direttamente al compratore finale, senza intermediari: vendita porta a
porta(prodotti che necessitano di una dimostrazione), telefono, posta, Internet, spaccio aziendale, ecc. \newline
Può essere indicato per le seguenti tipologie:
\begin{itemize}
	\item prodotti deperibili (es. prodotti ortofrutticoli)
	\item prodotti di alto livello qualitativo
	\item prodotti che richiedono molti servizi pre e post vendita
	\item prodotti “su misura” (es. abiti sartoriali)
	\item prodotti con alto valore unitario (es. aeroplani)
\end{itemize}

\paragraph{Canale indiretto breve} 
$ \\ Produttore \rightarrow Dettagliante \rightarrow Compratore finale \\$
Un agente di commercio interviene quando i prodotti sono standardizzati e la clientela è numerosa, e fa da connessione tra produttore e dettagliante.

\paragraph{Canale indiretto lungo}
$ \\ Produttore \rightarrow Grossista\rightarrow Dettagliante \rightarrow Compratore finale  \\$
Attraverso un agente di commercio il produttore vende a un grossista, il quale a sua volta vende all’utilizzatore finale.

\subsection{Intermediari}
Funzioni e vantaggi di usare intermediari:
\begin{itemize}
	\item \textbf{Informazione} Conoscono il consumatore e le ragioni per cui comprano e non comprano.
	\item \textbf{Stimolare la domanda} Favoriscono il contatto tra impresa e compratori.
	\item \textbf{Specializzazione}. Sono specializzati nel distribuire e garantiscono una maggiore efficienza.
	\item \textbf{Scorte} La produzione è programmata secondo certi ritmi. Gli intermediari consentono lo stoccaggio di una parte di prodotto.
	\item \textbf{Servizi post vendita} Mantengono il rapporto con il cliente mediante servizi post vendita.
	\item \textbf{Rischi} Possono assumere i rischi legati ad una domanda inferiore a quella prevista.
	\item \textbf{Facilitare gli scambi} Facilita il contatto tra produttore e
	consumatore.
\end{itemize}
Limiti e svantaggi di usare intermediari:
\begin{itemize}
	\item È più difficile ottenere l’esclusiva dai migliori distributori.
	\item Se i distributori vendono più prodotti tra loro concorrenti sono portati a spingere quelli che danno i margini più alti
	\item Se i distributori vendono più prodotti tra loro concorrenti possono essere riluttanti a collaborare al lancio di nuovi prodotti, in quanto preferiscono i prodotti di sicuro successo
	\item Determina margini e profitti economici inferiori
	\item Limita il rapporto diretto con il cliente finale
\end{itemize}

\section{Promozione o comunicazione}
Per realizzare gli obiettivi di marketing l’impresa si avvale di un complesso di strumenti di comunicazione al fine di fornire informazioni e nel contempo favorire l’atteggiamento e il comportamento di acquisto dei consumatori ed utilizzatori. \newline
Le principali forme di comunicazione (communication mix) sono:
\begin{itemize}
	\item pubblicità
	\item promozione delle vendite
	\item vendita personale
	\item pubbliche relazioni
	\item marketing diretto	
\end{itemize}

Oltre il 70\% delle aziende investe nella comunicazione multicanale per far parlare di sé e per farsi conoscere a nuovi target di riferimento.


\subsubsection{Tipi di promozione}
\begin{itemize}
	\item Promozione verso il consumatore finale, fatte dal produttore o dall'intermediario, mediante campioni gratuiti, coupon, premi, estrazioni ecc.
	\item Promozioni verso gli intermediari. Lo scopo è stimolare costoro	a promuovere a loro volta con efficacia il prodotto. Si tratta di sconti sulle quantità acquistate, campioni gratuiti, premi.
\end{itemize}



\subsection{Le emozioni}
Tutte le emozioni sono, essenzialmente, \textbf{impulsi ad agire}; in altre parole piani d’azione dei quali ci ha dotati l'evoluzione per gestire in tempo reale le emergenze della vita.
Il senso del movimento si rintraccia anche nell’etimo, che ci suggerisce il verbo latino “moveo”, ovvero “muovere”. È nella sua combinazione con il prefisso “e-“ che il lemma genera il suo significato di “muovere da” e ci conduce all’emozione come movimento da, come flusso di un agire che si sposta, che viaggia, che si genera e si sviluppa in un percorso da-a.
L’impulso ad agire viene favorito, a livello cerebrale, dalla presenza di \textbf{marcatori somatici}.
I marcatori somatici hanno la funzione di \textbf{‘registrare’ le nostre esperienze}, contrassegnandole in base alle sensazioni viscerali e non viscerali che proviamo in una data situazione. \textbf{Ci orientano} verso una gamma di alternative possibili di comportamenti, permettendoci di scegliere uno di essi: un marcatore somatico negativo funge da campanello d'allarme per il futuro, inibendo comportamenti futuri, un marcatore somatico positivo
incentiva comportamenti futuri. Ad esempio l’acquisto di un prodotto il cui uso è associato ad emozioni positive oppure il cui posizionamento o la cui comunicazione sono associati a valori che evocano emozioni positive, sono associati ad un marcatore somatico positivo, invece il morso di un cane e il dolore da esso causato è associato ad un marcatore negativo. L’informazione emotiva è trasmessa dai neuroni attraverso una variazione di frequenza dei loro impulsi. La ricerca neurologica ha dimostrato che i neuroni si accendono nello stesso modo e nelle stesse aree sia a seguito di un’azione compiuta dall’individuo sia a seguito di un’azione osservata, come il comportamento osservato nell’ambito di una comunicazione pubblicitaria. I neuroni che si accendono a seguito della semplice osservazione vengono chiamati \textbf{neuroni specchio}. L’esistenza di una forma di rispecchiamento, vale a dire la riproduzione all’interno di noi dello stato in cui si trovano gli altri, è alla base dell’empatia e dell’apprendimento, dell’identificazione e della comprensione delle intenzioni altrui oltre che del desiderio. Il marketing dunque può influire sugli atteggiamenti e sulle motivazioni che determinano il comportamento futuro degli acquirenti-consumatori, innescando, attraverso la comunicazione, un processo di apprendimento passivo, che prescinde dall’esperienza diretta del consumatore.
\paragraph{Il marketing esperienziale} Pone l’accento sulla centralità dell'esperienza d’uso e di consumo del cliente: è quello che un consumatore prova quando entra in contatto con un bene e un servizio e con la comunicazione di questi ultimi. 
Schmitt suddivide le esperienze in 5 tipologie, dette SEM(Moduli Strategici Esperienziali): 
\begin{itemize}
	\item \textbf{feel:} esperienze che suscitano sentimenti ed emozioni
	\item  \textbf{sense:} esperienze legate alla percezione sensoriale, la vista, l’udito, il tatto, il gusto e l’olfatto
	\item  \textbf{think:} esperienze che coinvolgono i processi cognitivi di apprendimento: hanno l’obiettivo di creare esperienze cognitive e di problem-solving
	\item  \textbf{act:} esperienze che spronano il consumatore ad agire, ad assumere determinati comportamenti e stili di vita
	\item  \textbf{relate:} esperienze derivanti da interazioni e relazioni sociali: vengono arricchite le esperienze individuali mettendo in relazione l’individuo con il suo sé ideale, con gli altri individui e con le altre culture. (es. «Think different» di Apple vuole far sentire i suoi clienti parte di un’élite sociale composta da ribelli e creativi)
\end{itemize}
Schmitt propone l’approccio di marketing esperienziale in
contrapposizione all’approccio di marketing classico.
L’autore, in particolare, mette in discussione l’impostazione razionale e utilitaristica tipica del marketing tradizionale che vede il consumatore come un soggetto razionale che decide in base alle caratteristiche e ai benefici funzionali dei prodotti. I consumatori, dice Schmitt, sono sempre più alla ricerca di esperienze che coinvolgano i sensi, il cuore e la mente; essi cercano prodotti, comunicazione e campagne di marketing con i quali relazionarsi e che possano incorporare nel loro stile di vita.

\subsection{Pubblicità}
La pubblicità è una delle componenti dell'azione di
marketing e il suo ruolo è inseparabile da quello degli altri fattori che concorrono alla vendita. In generale la pubblicità può essere efficace solo quando agiscono anche gli altri elementi del piano di marketing: prodotto differenziato, prezzo attraente e distribuzione sufficiente.

La pubblicità risponde ad un bisogno di \textbf{informazione} e si
rivela più utile quando l’acquirente si trova di fronte a prodotti con i quali ha \textbf{scarsa familiarità}, in particolari prodotti le cui caratteristiche non sono evidenti a colpo d’occhio.
Perché una comunicazione pubblicitaria sia veramente
efficace, deve mettere in luce una particolarità specifica, una \textbf{qualità distintiva}, un valore del prodotto che gli conferisca superiorità sui prodotti della concorrenza e che lo \textbf{posizioni nella mente dell'acquirente}. \newline
Obiettivi:
\begin{itemize}
	\item \textbf{Raggiungere un ampio numero} di potenziali compratori target
	\item \textbf{Informare} che un prodotto è disponibile
	\item \textbf{Convincere e persuadere:} orientare il consumatore verso un prodotto, facendolo preferire tra molti altri
	\item \textbf{Ricordare e rinforzare} un acquisto già effettuato
	\item \textbf{Migliorare i rapporti con gli intermediari} poiché offrono prodotti già noti ai consumatori
	\item \textbf{Rafforza l’immagine} di una marca o di un’impresa
\end{itemize}

\subsubsection{Tipi di pubblicità}
\paragraph{Pubblicità di prodotto} ha lo scopo di aumentare le vendite di prodotto rafforzando la
comunicazione del posizionamento scelto. Talvolta l’impresa è in secondo piano.
\paragraph{Pubblicità istituzionale}
ha l’obiettivo di creare un’immagine favorevole all'impresa nel suo complesso. Questo tipo di pubblicità può utilizzare i mezzi tradizionali oppure lo sponsorship: concerti, restauri, avvenimenti
sportivi. L’evento a cui si decide si associare il proprio nome deve essere sempre coerente con il proprio posizionamento.

\subsubsection{Pubblicità e ciclo di vita del prodotto}
Quando la domanda globale è espandibile la pubblicità ha più forte impatto sul mercato, contribuendo in particolare ad accelerare la diffusione del prodotto: la pubblicità svolge un ruolo di catalizzatore della domanda.

Quando il prodotto-mercato è in fase di maturità, la pubblicità ha un ruolo di mantenimento o influisce soprattutto sulle quote di mercato.

\subsubsection{Tecniche di persuasione}
\begin{itemize}
	\item \textbf{Dimostrazione:} è utilizzata per mostrare che cosa fa il prodotto. Più è spettacolare e incredibile, meglio è.
	\item \textbf{Scena di vita:} viene utilizzata per inserire il prodotto al centro di una storia quotidiana, in modo da aumentare il coinvolgimento del fruitore. L’obiettivo è di fare in modo che lo spettatore si identifichi nella situazione a lui familiare. Ha la funzione di: generare un atteggiamento di simpatia, emozione e calore verso il prodotto; mostrare come il prodotto funziona e risolve i problemi.
	\item \textbf{Sottointeso:} basata sull'intesa tra creativi e pubblico; questa tecnica afferma per accenni e rinvii, può servire anche per comunicare dichiarazioni comparative.
	\item \textbf{Ammiccamento:} spesso il riferimento è all’ambito sessuale.
	\item \textbf{Presupposizione:} consiste nel far credere all'ascoltatore che parte del messaggio che sta ricevendo sia scontata, senza bisogno di essere dimostrata.
	\item \textbf{Testimonianza}
	\begin{itemize}
		\item \textbf{Il personaggio famoso:} risulta efficace quando l’immagine del personaggio celebre è in qualche modo correlata al prodotto.
		\item \textbf{L’esperto:} celebre o meno, l’importante è che sia percepito come testimonial credibile.
		\item \textbf{La gente comune:} contrariamente a quanto si può pensare, questa	tipologia ha un’elevata credibilità. Molte interviste realizzate con normali consumatori risultano più efficaci di quelle realizzate con	attori, poiché questi ultimi vengono vissuti per quello che realmente sono, e cioè una finzione.
	\end{itemize}		
\end{itemize}

\subsubsection{Aspetti etici}
Dal punto di vista etico, una frequente critica che viene mossa, è che la pubblicità è in grado di far percepire delle differenziazioni inesistenti nonché consentire di ottenere prodotti con prezzi più elevati rispetto a coloro che non svolgono tale azione.

Dal punto di vista sociale, una frequente critica che viene mossa, è che la pubblicità induce al consumismo, spinge alla soddisfazione materialistica delle esigenze del consumatore ignorando gran parte dei valori dello spirito.

Questa critica va considerata insieme alla manipolazione dei desideri dei consumatori, una pubblicità cosiddetta di tipo subliminale
in cui vengono superate le barriere dell'apprendimento di tipo conscio, agendo direttamente sull'inconscio.

\subsection{Promozione delle vendite}
La promozione delle vendite è un’attività che riunisce un insieme di tecniche e mezzi di comunicazione, messi in atto nell'ambito del piano d’azione commerciale dell'impresa, allo scopo di suscitare nel target prescelto la nascita o l’evoluzione di un comportamento d’acquisto o di consumo a breve o a lungo termine.
\\
A differenza della comunicazione pubblicitaria la promozione opera principalmente sul comportamento del consumatore e meno sulla mente; infatti se gli acquirenti hanno notizia di saldi o vendite promozionali, attivano immediatamente comportamenti d’acquisto. \newline
Le tecniche promozionali vengono così individuate:
\begin{itemize}
	\item \textbf{Riduzioni di prezzo} Si tratta essenzialmente di offrire la stessa cosa ad un prezzo meno elevato, ricorrenti a diversi procedimenti: buono sconto, offerte speciali, 3x2, ritiro di prodotto, ecc
	\item \textbf{Vendita con premi ed omaggi} Agli acquirenti del prodotto vengono offerti gratuitamente piccoli oggetti, immediatamente o successivamente all'acquisto.
	\item \textbf{Prove e campioni} La distribuzione gratuita dei campioni o le degustazioni permettono ai consumatori di provare il prodotto.
	\item \textbf{Giochi e concorsi} Si tratta di gare a carattere ludico, che alimentano a speranza di vincite elevate.
	\item \textbf{Presentazione nel luogo di acquisto} Nei luoghi di vendita il produttore fornisce dimostrazioni circa le caratteristiche del prodotto.
	\item \textbf{Dimostrazioni} Particolarmente efficace per prodotti alimentari e cosmetici.
	\item \textbf{Fiere, mostre e seminari} Particolarmente efficaci per beni strumentali (macchinari) e beni di consumo durevole (mobili).
\end{itemize}

\subsection{Pubbliche relazioni}
Le pubbliche relazioni riuniscono le comunicazioni
elaborate dall'impresa per far conoscere l’esistenza, le azioni e le finalità dell'impresa e sviluppare
un’immagine favorevole nella mente del pubblico in
generale e degli interlocutori istituzionali e commerciali. Nelle pubbliche relazioni, l’obiettivo non è parlare del prodotto, bensì creare e consolidare un atteggiamento positivo nei confronti dell'impresa tra i vari segmenti di pubblico. \newline
Gli strumenti utilizzati dalle pubbliche relazioni possono essere riuniti in quattro categorie:
\begin{itemize}
	\item \textbf{Pubblicazioni}, come riviste aziendali, rapporti annuali, opuscoli per la clientela, cataloghi che al giorno d’oggi sono spesso disponibili sul sito Internet dell'impresa.
	\item \textbf{Eventi o manifestazioni}, quali gare sportive, concerti o	esposizioni sponsorizzate dall’impresa.
	\item \textbf{Informazioni relative all’impresa}, come il lancio di un nuovo prodotto, un contratto importante, un risultato nella R\&S (ricerca e sviluppo). Spetta agli
	esperti di pubbliche relazione organizzare gli incontri con i giornalisti e preparare il comunicato stampa.
	\item \textbf{Mecenatismo} o partecipazione dell'impresa a cause di interesse generale, umanitarie, scientifiche o culturali.
\end{itemize}
Kotler riassume le P.R. nella Parola “PENCILS”: pubblicazioni, eventi,notizie, attività comunitarie, simboli di identità, lobbying e responsabilità sociale.

\subsection{Vendita personale}
Uno degli strumenti di comunicazione di marketing più costoso è costituito dal personale di vendita dell'azienda.
Il personale di vendita ha il vantaggio di essere più efficace rispetto alla pubblicità. Il venditore stabilisce un contatto diretto con il cliente, può condurlo a pranzo, stimolare l’attenzione e l’interesse anche attraverso la fruizione del prodotto. Più il prodotto è complesso e maggiore è la necessità di impiegare personale diretto.

\subsection{Marketing diretto}
Tale forma di marketing, attraverso l’ausilio di uno o più mezzi di comunicazione, si propone di stabilire un dialogo con il mercato a livello di singolo cliente e non di gruppo o massa di clienti.

Gli obiettivi che si vogliono cogliere con l’azione di marketing diretto sono:
\begin{itemize}
	\item creare liste (data base) di potenziali clienti
	\item generare contatti con probabili clienti
	\item indurre al riacquisto i clienti già acquisiti facendo loro pervenire proposte mirate (ad esempio, individuando il nome di coloro che hanno acquistato un pc di recente si può pensare di proporre uno scanner)
\end{itemize}
Il marketing diretto trova oggi una sua espressione attraverso l’uso del social media marketing (Facebook, YouTube, Twitter e LinkedIn).

\chapter{Controllo}
Lo stadio conclusivo del piano di marketing è costituito dal controllo. È importante ricordare che le imprese di successo sono
imprese che apprendono; raccolgono i feedback che provengono dal mercato; analizzano e valutano i risultati; apportano le correzioni necessarie a migliorarli.

L’attività di controllo consente di verificare se l’eventuale mancato raggiungimento degli obiettivi di marketing risiede in una delle 4
P, oppure in un errore di segmentazione, targeting, posizionamento o definizione del bisogno mediante la prima fase della ricerca.
Cruciale per l’attività del controllo è – anche per questa fase – la “ricerca di mercato”, mediante il coinvolgimento
di panel di consumatori, monitorati, attraverso più rilevazioni (marketing relazionale), nel tempo.

Un’efficace azione di marketing si basa sul principio cibernetico di pilotare un’imbarcazione attraverso un costante monitoraggio della sua posizione in relazione della sua destinazione.

\end{document}





























