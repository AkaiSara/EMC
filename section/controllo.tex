\chapter{Controllo}
Lo stadio conclusivo del piano di marketing è costituito dal controllo. È importante ricordare che le imprese di successo sono
imprese che apprendono; raccolgono i feedback che provengono dal mercato; analizzano e valutano i risultati; apportano le correzioni necessarie a migliorarli.

L’attività di controllo consente di verificare se l’eventuale mancato raggiungimento degli obiettivi di marketing risiede in una delle 4
P, oppure in un errore di segmentazione, targeting, posizionamento o definizione del bisogno mediante la prima fase della ricerca.
Cruciale per l’attività del controllo è – anche per questa fase – la “ricerca di mercato”, mediante il coinvolgimento
di panel di consumatori, monitorati, attraverso più rilevazioni (marketing relazionale), nel tempo.

Un’efficace azione di marketing si basa sul principio cibernetico di pilotare un’imbarcazione attraverso un costante monitoraggio della sua posizione in relazione della sua destinazione.