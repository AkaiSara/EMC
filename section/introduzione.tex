\chapter{Introduzione}
\section{Nascita del Marketing}
\subsubsection*{Fase 1: Orientamento al prodotto: si vende cosa si può produrre}
Nella fase iniziale dello sviluppo industriale gli acquirenti erano ricettivi e poco esigenti e il mercato era caratterizzato da una scarsità di beni e servizi. L'attenzione delle aziende era quindi rivolta alle fasi tecnico-ingegneristiche di progettazione e fabbricazione del prodotto.

\subsubsection*{Fase 2: Orientamento alla vendita}
In questa fase si sviluppano i mercati di massa e l'offerta supera nel suo complesso la domanda. In tale contesto diventa critico il ruolo della funzione vendite rispetto alla fase precedente, in quanto si avverte la concorrenza delle altre aziende, si cerca quindi di allargare le proprie vendite sottraendo quote di mercato ai concorrenti. \newline
Si fa strada il convincimento che con adeguati investimenti in attività pubblicitarie e promozionali possa essere ampliata la domanda di qualsiasi prodotto. Ne conseguono strategie di vendita aggressive e di massa.

\subsubsection*{Fase 3: Orientamento al mercato: si produce cosa si può vendere}
La vendita "aggressiva" si dimostra insufficiente per sostenere il lancio di un prodotto non in sintonia con la domanda. L'attenzione, dunque, si sposta dalla vendita ai bisogni del consumatore. Bisogna orientare la produzione in funzione dei mutevoli bisogni dei consumatori spesso influenzati dalla moda.\newline
Con l'adozione dell'orientamento al mercato delle aziende nasce il marketing.

\section{Che cos'è il marketing}
\subsection{Definizione}
Il marketing rappresenta l’insieme delle attività mediante le quali un’organizzazione mira a soddisfare le esigenze di persone rendendo loro disponibili prodotti o servizi, sostenendo idee o affermando valori nella società.

\subsection{Bisogni VS Desideri}
Il bisogno è la mancanza totale o parziale di uno o più elementi che costituiscono il benessere della persona. I bisogni non sono creati dalla società o dal marketing; essi esistono nella sostanza stessa della biologia e della condizione umana. (vedere Piramide di Maslow).
Il desiderio è un sentimento intenso che spinge a cercare il possesso, il conseguimento o l’attuazione di quanto possa appagare un proprio bisogno. \newline
Mentre i bisogni sono relativamente pochi i desideri sono molteplici. I desideri umani vengono continuamente modellati e rimodellati. \newline
Il marketing, dunque, assieme ad altri agenti sociali, può solo influenzare e creare i desideri e non i bisogni.

\subsection{Cosa non è}
Il marketing non può essere considerato equivalente alla vendita in quanto ha inizio molto prima che un prodotto sia reso disponibile. Il marketing è l’attività che i manager devono svolgere al fine di identificare i bisogni attuali e potenziali, misurare la loro intensità e determinare se si presentano opportunità profittevoli. La vendita ha luogo solo dopo che un prodotto è stato realizzato. \newline
Il marketing non è una specifica funzione dell’impresa, è troppo importante per essere affidato esclusivamente alla funzione commerciale , ma è un orientamento culturale che guida l’insieme delle attività organizzative.

\subsection{Tipi di marketing}
\subsubsection*{Marketing dei beni e dei servizi} 
Riguarda gli scambi tra le imprese e i consumatori. Business To Consumer (B2C)
\subsubsection*{Marketing industriale} 
Riguarda gli scambi tra imprese e altre imprese. Business To Business (B2B)
\subsubsection*{Marketing sociale}
Il marketing sociale è l’insieme di attività intraprese da un’organizzazione per creare , mantenere e modificare gli atteggiamenti e i comportamenti collettivi in vista di valori socialmente condivisi. \newline
Il marketing sociale affronta temi sociali di carattere non controverso, quindi ad esempio la politica essendo controversa non è marketing sociale. \newline
Il marketing sociale spesso riguarda le organizzazioni senza fini di lucro. \newline
I principali settori in cui opera il marketing sociale riguardano
\begin{itemize}
	\item  la promozione e la tutela della salute (es. diffusione di abitudini alimentari e di stili di vita corretti)
	\item  la prevenzione e la riduzione di comportamenti a rischio (es. assunzione di fumo, alcool, droghe, comportamenti sessuali a rischio, comportamenti scorretti di guida, dipendenze comportamentali)
	\item la promozione di comportamenti sociali (es.risparmio energetico, riduzione inquinamento atmosferico, utilizzo del trasporto pubblico, raccolta  differenziata, donazione di sangue).
\end{itemize}

\subsection{Livelli di realizzazione del marketing}
\subsubsection*{Marketing di risposta}
Questa forma di marketing si manifesta quando esistono dei bisogni chiaramente definiti, con riferimento ai quali un certo numero di imprese si impegna a predisporre soluzioni valide.

\subsubsection*{Marketing di anticipo}
Questa forma di marketing riconosce un bisogno emergente o latente. È più rischioso del marketing di risposta: le imprese possono entrare sul mercato o troppo presto o troppo tardi, oppure possono commettere errori di valutazione ritenendo che un certo mercato possa svilupparsi.

\subsubsection*{Creazione di prodotti/servizi non richiesti}
Il marketing raggiunge il massimo livello quando un’impresa (imprese market-driving) introduce un prodotto o un servizio in precedenza non richiesto o immaginato da alcuno.

\subsection{Conoscenza degli altri operatori di mercato}
Conoscenza di influenzatori, intermediari e concorrenti.

\section{Opportunità di mercato}
\textit{Il marketing è l’arte di individuare, sviluppare e ricavare un profitto dalle opportunità. (Kotler, 1999) } \newline
Le situazioni che possono determinare il sorgere di opportunità di mercato sono principalmente tre:
\begin{itemize} 
	\item possibilità di fornire un prodotto o un servizio a offerta scarsa;
	\item possibilità di fornire un prodotto o un servizio esistente in modo nuovo o migliore;
	\item possibilità di fornire un prodotto o un servizio nuovi per il mercato.
\end{itemize}

Per capire verso cosa indirizzarsi si può chiedere agli utilizzatori di prodotti o servizi se riscontrano problemi nell’impiego dei prodotti in questione e se hanno dei suggerimenti da fare; oppure creare all’interno dell’azienda un sistema di raccolta che faccia confluire le nuove idee verso un punto centrale dove possono essere classificate, analizzate e valutate.

\section{Pensiero creativo}
Per progredire e quindi stimolare l’innovazione, non bastano i fondi, le risorse finanziarie, ma occorre, in pari misura, un’adeguata espressione e applicazione di quello che viene chiamato “pensiero divergente”.

\subsubsection*{Pensiero convergente}
Questo tipo di pensiero è caratterizzato dalla ripetizione del già appreso e dall’adattare vecchie risposte a situazioni nuove in modo più o meno meccanico.

\subsubsection*{Pensiero divergente}
Il pensiero divergente implica “fluidità, flessibilità, originalità” e riguarda essenzialmente la produzione di idee nuove e numerose.

\subsection*{Stimolare la creatività}
Esistono varie tecniche per stimolare la creatività. \newline
Alla base della generazione di idee vi è la  necessità di non dare nulla per scontato, nulla per vietato, nulla per immutabile. Cartesio suggeriva di non accogliere mai nulla per vero che non si conoscesse essere tale per evidenza. Analogamente, nessuna idea dovrebbe essere assurda finché questo non è dimostrato.

\subsubsection*{Tecnica dei sei cappelli}
È una tecnica ideata da Edward De Bono, attualmente considerato il massimo esponente nel campo del pensiero creativo. \newline
La tecnica si basa sulla convinzione che affrontare un problema globalmente, utilizzando tutte le capacità del nostro cervello contemporaneamente, produca scarsi risultati in termini creativi. De Bono osserva che, quando si pensa ad un determinato argomento si possono assumere sei diversi atteggiamenti mentali. Con la tecnica dei sei cappelli si deve portare un cappello alla volta, quindi si è costretti a passare attraverso tutti gli atteggiamenti separatamente, in sequenza, senza giungere affrettatamente alle conclusioni. La tecnica dei sei cappelli consente di pensare e dire cose che, altrimenti, metterebbero a rischio il nostro ego e di concentrare l’attenzione su un aspetto per volta del problema. Inoltre, i sei cappelli per pensare permettono di chiedere agli altri di assumere un certo modo di pensare come se si trattasse di una parte da recitare.
\begin{itemize}
	\item Cappello nero. Il cappello nero rappresenta la razionalità negativa: si tratta di esprimere ed evidenziare con obiettività gli elementi deboli, rischiosi o che destinano all’insuccesso un certo prodotto o progetto.
	\item Cappello giallo. Il giallo è un colore solare e positivo: rappresenta l’ottimismo, gli aspetti positivi, ossia tutte le ragioni per cui una soluzione funzionerà, fondandosi comunque sulla realtà
	\item Cappello verde. Il verde evoca l’immagine della natura, di crescita fertile: è il cappello creativo, delle nuove idee, proposte e alternative. Si tratta di ricercare altre modalità di soluzione, di produrre miglioramenti a soluzioni già individuate. De Bono suggerisce un modo per stimolare questa crescita: esporre idee provocatorie per poter uscire dagli schemi.
	\item Cappello bianco. Il bianco è un colore neutro, rappresenta i fatti e i dati oggettivi. “pensare con il cappello bianco”, afferma De Bono. “diventa una disciplina che aiuta il pensatore a separare nettamente i fatti da estrapolazioni o interpretazioni”.
	\item Cappello rosso. Il rosso suggerisce ira, rabbia ed emozioni, perciò indossare il cappello rosso significa che si devono esprimere le emozioni, i sentimenti, le sensazioni sia razionali che irrazionali senza doverne spiegare il perché.
	\item Cappello blu. Questo cappello rappresenta il controllo su tutto il processo: è con esso che si stabilisce quali altri cappelli occorre indossare e quando. Il cappello blu organizza il processo di pensiero.
\end{itemize}

\subsubsection*{Brainstorming}
Il brainstorming, letteralmente “tempesta del cervello”, è stato proposto verso il 1938 come metodo di creatività da Alex Osborn. Si tratta di una tecnica basata su una discussione di gruppo incrociata, guidata da un moderatore che mira a fare esprimere, in maniera assolutamente non univoca, il maggior numero possibile di idee su un determinato problema (Osborn, 1992). \newline
La constatazione su cui poggia questa tecnica è che le idee, una vota espresse, richiamano altre idee, e queste a loro volta, altre ancora. Un flusso continuo che si autoalimenta grazie all’apporto di tutti. Entrando in una sessione di brainstorming, ogni partecipante viene invitato ad abbandonare la logica ed i pregiudizi. Un gruppo di brainstorming si compone, generalmente, di dieci-dodici persone, opera sotto la guida di un animatore, agisce in un tempo limite di 45-50 minuti.

\section{Il piano di Marketing}
Il piano di marketing rappresenta il complesso di azioni coordinate che un’impresa realizza per raggiungere i propri obiettivi di marketing, nonché la sequenza di fasi e i tempi necessari per realizzarli. \newline
Un piano di marketing si compone delle seguenti fasi:
\begin{itemize}  
	\item Ricerca di mercato
	\item Segmentazione, targeting e posizionamento 
	\item Definizione e implementazione del marketing mix
	\item Controllo e valutazione dei risultati
\end{itemize}
