\chapter{Marketing mix}
Il marketing mix è la combinazione delle variabili controllabili di marketing che l’impresa impiega al fine di conseguire gli obiettivi stabiliti nell’ambito del mercato di riferimento. Definire un marketing mix significa sviluppare una proposta di valore complessiva attraverso la combinazione e l’integrazione di alcuni fattori critici. \newline
Le 4 macrovariabili che costituiscono il marketing mix sono:
\begin{enumerate}
	\item prodotto (product)
	\item prezzo (price)
	\item punto vendita e distribuzione (place)
	\item promozione (promotion)
\end{enumerate}

Alle 4 P del marketing mix si tende oggi ad  aggiungerne una quinta, di fondamentale rilevanza: il personale. Il fattore umano è diventato determinante per le attività di marketing, soprattutto nell'ambito dei servizi. 

\section{Prodotto}
Si definisce prodotto qualsiasi bene o servizio scambiabile sul mercato che può rispondere alle esigenze di un compratore. È tutto quello che il compratore considera quando decide per l’acquisto. \newline
Un servizio è qualsiasi attività o vantaggio che una parte può scambiare con un’altra, la cui natura sia essenzialmente intangibile e non implichi proprietà di alcunché. \newline
Le caratteristiche distintive del servizio sono:
\begin{itemize}
	\item l’intangibilità
	\item la contemporaneità di erogazione e fruizione
	\item il coinvolgimento diretto del fruitore nella sua realizzazione
	\item la variabilità, dovuta alla componente personale
	\item la deperibilità, poiché i servizi non si possono restituire/rivendere
\end{itemize}
L’idea che sta alla base del concetto di prodotto è che i consumatori non acquistano solo aspetti fisici, ma ricercano soprattutto un mezzo per soddisfare le loro esigenze e i loro desideri. In sostanza i consumatori ricercano benefici: le persone non comprano le cose solo per quello che servono, ma anche per quello che significano. \newline
Compito dell’operatore di marketing è di scoprire i benefici attesi dal consumatore e di offrirglieli, e non solo offrire attributi. \newline
Classificazione dei prodotti:
\begin{itemize}
	\item \textbf{Beni destinati alla produzione}: prodotti/servizi acquistati per essere impiegati nella produzione di altri beni/servizi.
	\item \textbf{Beni di consumo}: utilizzati dai consumatori/fruitori finali.
	\begin{itemize}
		\item Beni durevoli : producono la loro utilità nel tempo.
		\item Beni non durevoli : esauriscono la loro utilità in un’unica
		soluzione o in pochi usi ripetuti (alcool test, penne, ...).
		\item Beni di largo consumo ad acquisto frequente ( prodotti alimentari, sigarette ).
		\item Beni di largo consumo con forte preferenza di marca
		( abbigliamento, arredamento ).
		\item Beni ad acquisto saltuario e ponderato (abitazioni).
		\item Beni speciali (beni di lusso)
	\end{itemize}
    \item \textbf{Servizi}: prestazioni derivanti da una attività (di persone o di organizzazioni) o dalla disponibilità temporanea di un prodotto.
\end{itemize}
Attributi del prodotto:
\begin{itemize}
	\item forma, colore, design;
	\item qualità;
	\item dimensione, peso, ingombro;
	\item packaging;
	\item marca;
	\item servizi ai clienti;
	\item garanzia.
\end{itemize}
La forma di un prodotto e più in generale il suo design – ossia lo stile che caratterizza gli oggetti e che riguarda tanto gli aspetti estetici , quanto il colore e i materiali – rappresenta uno degli elementi più significativi che contraddistinguono non sono i prodotti di consumo, ma anche i beni destinati alla produzione. \newline
Il design consente di differenziare i prodotti ed è quindi legato anche al comportamento psicologico del compratore. \newline
Il design non è soltanto un attributo di carattere estetico; spesso le imprese studiano in modo approfondito il design per migliorare l’efficienza nella produzione, per rendere più facile l’utilizzazione del prodotto da parte del consumatore e per rendere il prodotto più sicuro. \newline
L’importanza del design non si limita al prodotto in sé, ma riguarda anche la sua presentazione cioè l’esposizione della merce nel negozio, attraverso l’uso del “display”("lineare espositivo"  in italiano, cioè il supporto su cui viene esposto il prodotto). \newline
Di estrema importanza nel design dei prodotti è anche l’uso del colore. I colori caldi favoriscono dunque l’acquisto d’impulso, mentre i colori freddi (blu e verde) sono più rilassanti e più adatti all’acquisto di beni durevoli. 

\subsection{Marca}
Per marca intendiamo un nome, un termine, un simbolo, un design o una combinazione di questi, che mira ad identificare i beni o servizi di un’impresa e a differenziarli da quelli di altre imprese.
La marca consente di esprimere le prestazioni tecniche di un prodotto, la forza dell’azienda, il modo con cui l’impresa fa impresa, la sua essenza profonda e i suoi valori. \newline
Un elemento fondamentale della marca è la “reputazione dell’azienda”. La comunicazione deve spingersi a comunicare ciò che c’è dietro “le quinte” del prodotto. \newline
La marca deve comunicare la stabilità e la sicurezza del prodotto e dell’azienda. \newline
La storia della marca contribuisce ad aumentare il valore. La marca e il suo vissuto rassicurano il cliente sulla validità della scelta. \newline
Principali funzioni della marca:
\begin{itemize}
	\item comunicazione e informazione, non solo nei confronti dei clienti ma di tutti gli interlocutori dell’impresa, esterni e interni;
	\item conferimento di un posizionamento al prodotto;
	\item differenziazione;
	\item conferimento di identità all’impresa;
	\item evocazione di immagini e percezioni;
	\item promessa di valore;
	\item garanzia di qualità.
\end{itemize}
Quando parliamo di marca ci possiamo riferire a varie tipologie della stessa. Un’importante distinzione è quella tra marca del produttore e marca del venditore. La prima è la marca apposta dal produttore del prodotto (es. Cirio) e viene denominata marca industriale. La seconda è quella apposta dal venditore del prodotto: in genere, si tratta di una catena di supermercati (es. Coop) o di grandi magazzini che pongono in vendita dei prodotti realizzati da altre imprese con il proprio nome.

\subsection{Packaging}
La confezione da semplice strumento di imballo del prodotto utile a fini di conservazione e di movimentazione è divenuta, per molti prodotti, un importante strumento di marketing, sia per il consumatore per il quale svolge importanti funzioni di individuazione del prodotto, sia per i distributori per i quali svolge funzioni espositive e promozionali.
\begin{itemize}
	\item \textbf{Imballo primario}: rappresenta l’essenziale contenitore che trattiene il prodotto e che ne consente la conservazione.
	\item \textbf{Imballo secondario}: si riferisce a contenitori addizionali che possono essere aggiunti a fini protettivi o per esigenze di marketing.
	\item \textbf{Confezione display}: costituita dal packaging necessario per esporre il prodotto sul punto vendita.
	\item \textbf{Confezione di stoccaggio (terzario)}: necessaria per trasportare la merce o immagazzinarla nei magazzini.
\end{itemize}

L'imballaggio o "packaging" dei prodotti rappresenta una delle principali cause della produzione dei rifiuti, con conseguente problema di smaltimento.

\subsection{Servizi ai clienti}
I confini tra prodotti e servizi tendono a diventare sempre più sfumati e dinamici.
I classici servizi di assistenza ai clienti pre e post vendita sono:
\begin{itemize}
	\item servizio informazioni;
	\item servizio credito;
	\item servizio manutenzione;
	\item servizio reclami.
\end{itemize}

La tendenza attuale è verso una sempre maggiore personalizzazione dei servizi, grazie alla gestione di un data-base contenente informazioni sui singoli clienti.

\subsection{Garanzia e certificazioni}
La garanzia rappresenta un altro importante attributo e strumento di differenziazione del
prodotto dalla concorrenza. Ci sono due tipi di garanzia:
\begin{itemize}
	\item \textbf{Garanzia implicita:} la capacità del prodotto di svolgere la funzione promessa e di non recare rischi per la sicurezza della persona.
	\item \textbf{Garanzia esplicita e certificazioni di qualità:} comunicano al cliente un determinato livello qualitativo del prodotto e gli forniscono un servizio di assistenza valido entro un certo arco temporale.
\end{itemize}

\subsection{Qualità del prodotto}
Il cliente valuta la qualità del prodotto:
\begin{itemize}
	\item \textbf{all’atto della scelta:} confrontando i benefici attesi e quelli percepiti dagli attributi tangibili ed intangibili.
	\item \textbf{nell’uso/fruizione del prodotto/servizio:} questo comporta la conferma o la modifica del giudizio di	qualità precedentemente espresso.
\end{itemize}
Fattori su cui il cliente fonda il proprio giudizio:
\begin{itemize}
	\item fisico-strutturali (packaging, design, tecnologia)
	\item funzionali (prestazioni, sicurezza, facilità d’uso)
	\item di servizio (pre/post vendita, accessibilità, reperibilità)
	\item psicologici (estetica, design, moda, comunicazione)
	\item di immagine (del prodotto e della marca)
\end{itemize}

\subsubsection{Soddisfazione del cliente}
Il \textit{modello Servqual} di Parasuraman, Zeithaml e Berry si propone di fornire una misura della qualità percepita di un servizio (customer satisfaction), e quindi della
soddisfazione del cliente, attraverso un confronto tra:
\begin{itemize}
	\item le aspettative/attese con cui il cliente si accosta alla tipologia di
	servizio
	\item le percezioni del servizio avvenute dopo il consumo/utilizzo
\end{itemize}

Questo confronto è operato attraverso una metodologia detta paradigma della discrepanza. Essa si basa su un criterio sottrattivo tra livello delle percezioni di un prodotto/servizio con il livello di aspettative in relazione a quella tipologia di prodotto/servizio. La soddisfazione è intesa come stato psicologico derivante da un gap tra la valutazione dell'avvenuta esperienza di consumo e le attese del consumatore in merito a tale esperienza.

\subsection{Ciclo di vita}
Il ciclo di vita descrive l’andamento delle vendite di un prodotto nel tempo attraverso un modello ideale: una curva crescente e poi decrescente.

Tipologie di ciclo di vita:
\begin{itemize}
	\item prodotto a vita lunga (introduzione, crescita, maturità)
	\item prodotto di moda (introduzione e crescita)
	\item prodotto gadget (introduzione)
\end{itemize}

\subsubsection{Introduzione}
Inizia quando il prodotto viene introdotto nel sistema produttivo. In questa fase viene stimolata la “domanda primaria” che mira a convincere i consumatori circa i benefici che possono ottenere dall’acquisto del nuovo prodotto. I prezzi in genere sono alti, ma lo sono anche i costi di distribuzione e di promozione. Le vendite crescono molto lentamente e i profitti sono modesti.
Lo scopo dell’impresa è creare la domanda globale il più rapidamente possibile per uscire dalla fase di incertezza.
 
Lo scopo viene raggiunto tramite il conseguimento di questi obiettivi:
\begin{itemize}
	\item Rendere nota l’esistenza del prodotto
	\item Informare il mercato sui vantaggi dell’innovazione
	\item Incoraggiare gli acquirenti a provare il nuovo prodotto
	\item Introdurre il prodotto nella rete di distribuzione
\end{itemize}

\subsubsection{Sviluppo}
Se i primi consumatori, dopo aver adottato il prodotto ripetono i loro acquisti perché sono soddisfatti, le vendite cominciano a crescere . I prezzi restano alti poiché la domanda
è forte; i costi di produzione e di distribuzione sono ora ripartiti su volumi elevati di vendita determinando una crescita dei profitti. La domanda in espansione facilita l’entrata nel mercato di nuove imprese aumentando la concorrenza. 
Lo scopo dell’impresa è ampliare ed estendere il mercato, dato che la domanda è espandibile.

Lo scopo viene raggiunto tramite il conseguimento di questi obiettivi:
\begin{itemize}
	\item Adottare un sistema di distribuzione intensivo
	\item Migliorare il prodotto, aggiungendo nuove caratteristiche
	\item Rafforzare il sistema di comunicazione basato sul posizionamento prescelto
\end{itemize}

\subsubsection{Maturità}
La domanda rallenta ulteriormente. Il numero dei concorrenti tende ad essere molto elevato, determinando l’uscita dal mercato delle imprese più deboli. I profitti cominciano a
calare poiché i prezzi sono spinti verso il basso dalla competizione. In questa fase le imprese aumentano le spese di ricerca e sviluppo allo scopo di migliorare i prodotti, nonché quelle di promozione e di pubblicità.
Lo scopo dell’impresa è mantenere e ampliare la quota di mercato conquistando un vantaggio competitivo difendibile sui concorrenti.

Lo scopo viene raggiunto tramite il conseguimento di questi obiettivi:
\begin{itemize}
	\item Migliorare le qualità del prodotto e allargare le sue prestazioni
	\item Esplorare nuove nicchie
	\item Adottare un marketing relazionale one to one che pone l’accento sulla soddisfazione a lungo termine della clientela allo scopo di creare e mantenere la fedeltà dei clienti esistenti
\end{itemize}

\subsubsection{Declino}
Le vendite cominciano a diminuire in seguito al cambiamento dei gusti del consumatore, all’introduzione di nuovi prodotti sostitutivi e all’innovazione tecnologica. In questa fase generalmente l’organizzazione non investe più, poiché i costi sono superiori ai ricavi.

\section{Prezzo}
\subsection{Per il consumatore}
Al prezzo corrisponde il valore percepito del prodotto, che è
in relazione:
\begin{itemize}
	\item alle caratteristiche funzionali (qualità, prestazioni)
	\item alle valenze psicologiche (immagine associata al prodotto e alla marca)
	\item ai servizi (garanzia, assistenza pre e post vendita, condizioni di pagamento)
\end{itemize}
Perché il consumatore acquisti un dato prodotto è necessario che il prezzo ricada entro un certo intervallo, che può variare nel tempo, per le diverse circostanze e sulla base degli stimoli che riceve dall’impresa.
Se il prezzo è troppo basso il consumatore diventa sospettoso: potrebbe essere merce falsa o rubata, se è troppo alto decide che ìl prezzo è superiore al valore che attribuisce al prodotto.

\subsection{Per l'impresa}
Il prezzo è una variabile di marketing mix.
Per l’impresa esistono almeno tre criteri per determinare il
prezzo:
\begin{enumerate}
	\item il costo di produzione
	\item l’andamento della domanda
	\item il comportamento della concorrenza
	%\item canale di distribuzione
\end{enumerate}
Il prezzo deve permettere all'impresa il recupero dei costi
sostenuti e il conseguimento dell'utile.
I ricavi e i costi hanno andamenti diversi. Esiste un punto di equilibrio chiamato “break-even point” dove costi e ricavi sono uguali. Questo punto è determinato dal numero di prodotti venduti: al di sotto di tale numero si producono perdite, al di sopra di realizza utile.

\paragraph{1)Strategia di prezzo basata sui costi}
Il costo rappresenta un livello al di sotto del quale l’impresa non può scendere per un periodo di tempo lungo senza compromettere non soltanto la redditività, ma anche la sopravvivenza.
Nel definire i prezzi molte imprese aggiungono un “ricarico” alle stime di costo. Tale metodo (cost plus pricing) consiste nel determinare il prezzo sommando al costo una quota di profitto sperato (es. ai costi aggiunto il 10\%).

\paragraph{2)Strategia di prezzo basata sulla domanda}
Determinazione del prezzo sulla base del valore. Viene stimato il prezzo massimo che il consumatore è disposto a pagare per il prodotto. Viene fissato un prezzo inferiore, in modo da lasciare un “surplus al consumatore”.
Le situazioni che si possono presentare sono molte, ma si
possono ridurre a due:
\begin{itemize}
	\item la domanda è bassa, ma si presume che possa successivamente aumentare quando il consumatore abbia avuto modo di conoscere il prodotto.
	\item la domanda è elevata, ma si presume che debba diminuire o per saturazione del mercato o perché la concorrenza riuscirà a sottrarre quote di mercato.
\end{itemize}

\paragraph{3)Strategia di prezzo basata sulla concorrenza}
Se un’impresa è in posizione di leadership per dimensione,
caratteristiche del prodotto, può entro certi limiti, ignorare il comportamento dei concorrenti e decidere di cambiare i prezzi (i prezzi degli pneumatici sono fissati dalla Michelin).
Se invece i concorrenti sono agguerriti e l’impresa è di piccole
dimensioni, il livello del proprio prezzo rispetto a quello della concorrenza va attentamente valutato (al livello di mercato o al di sotto dei livelli di mercato).

\subsubsection{Le politiche di prezzo} In generale esse variano nelle fasi di vita del prodotto: prezzi alti nella fase di introduzione, prezzi bassi per consentire l’estensione nelle fasi successive. 
Differenti segmenti di mercato rispondono a differenti prezzi di linee di prodotto.
Il servizio ai clienti è correlato al prezzo; prezzi bassi in genere sono associati a un modesto servizio al cliente.

\section{Punto vendita e distribuzione}
Per sistema di distribuzione fisica o logistica si intende il coordinamento delle attività dell’impresa che mirano a trasferire materialmente i prodotti al consumatore finale nei tempi attesi.
\paragraph{Politica distributiva} Le decisioni e le azioni dell’impresa relative ai canali di distribuzione, alle scelte e alla valutazione degli intermediari commerciali, all’organizzazione e alla gestione della forza di vendita.

\paragraph{Canale di distribuzione}
L’insieme di persone e di istituzioni che svolgono le funzioni necessarie per trasferire il prodotto
dal produttore al cliente.

\subsection{Classificazione dei canali distributivi}
\paragraph{Canale diretto}
Il produttore vende direttamente al compratore finale, senza intermediari: vendita porta a
porta(prodotti che necessitano di una dimostrazione), telefono, posta, Internet, spaccio aziendale, ecc. \newline
Può essere indicato per le seguenti tipologie:
\begin{itemize}
	\item prodotti deperibili (es. prodotti ortofrutticoli)
	\item prodotti di alto livello qualitativo
	\item prodotti che richiedono molti servizi pre e post vendita
	\item prodotti “su misura” (es. abiti sartoriali)
	\item prodotti con alto valore unitario (es. aeroplani)
\end{itemize}

\paragraph{Canale indiretto breve} 
$ \\ Produttore \rightarrow Dettagliante \rightarrow Compratore finale \\$
Un agente di commercio interviene quando i prodotti sono standardizzati e la clientela è numerosa, e fa da connessione tra produttore e dettagliante.

\paragraph{Canale indiretto lungo}
$ \\ Produttore \rightarrow Grossista\rightarrow Dettagliante \rightarrow Compratore finale  \\$
Attraverso un agente di commercio il produttore vende a un grossista, il quale a sua volta vende all’utilizzatore finale.

\subsection{Intermediari}
Funzioni e vantaggi di usare intermediari:
\begin{itemize}
	\item \textbf{Informazione} Conoscono il consumatore e le ragioni per cui comprano e non comprano.
	\item \textbf{Stimolare la domanda} Favoriscono il contatto tra impresa e compratori.
	\item \textbf{Specializzazione}. Sono specializzati nel distribuire e garantiscono una maggiore efficienza.
	\item \textbf{Scorte} La produzione è programmata secondo certi ritmi. Gli intermediari consentono lo stoccaggio di una parte di prodotto.
	\item \textbf{Servizi post vendita} Mantengono il rapporto con il cliente mediante servizi post vendita.
	\item \textbf{Rischi} Possono assumere i rischi legati ad una domanda inferiore a quella prevista.
	\item \textbf{Facilitare gli scambi} Facilita il contatto tra produttore e
	consumatore.
\end{itemize}
Limiti e svantaggi di usare intermediari:
\begin{itemize}
	\item È più difficile ottenere l’esclusiva dai migliori distributori.
	\item Se i distributori vendono più prodotti tra loro concorrenti sono portati a spingere quelli che danno i margini più alti
	\item Se i distributori vendono più prodotti tra loro concorrenti possono essere riluttanti a collaborare al lancio di nuovi prodotti, in quanto preferiscono i prodotti di sicuro successo
	\item Determina margini e profitti economici inferiori
	\item Limita il rapporto diretto con il cliente finale
\end{itemize}

\section{Promozione o comunicazione}
Per realizzare gli obiettivi di marketing l’impresa si avvale di un complesso di strumenti di comunicazione al fine di fornire informazioni e nel contempo favorire l’atteggiamento e il comportamento di acquisto dei consumatori ed utilizzatori. \newline
Le principali forme di comunicazione (communication mix) sono:
\begin{itemize}
	\item pubblicità
	\item promozione delle vendite
	\item vendita personale
	\item pubbliche relazioni
	\item marketing diretto	
\end{itemize}

Oltre il 70\% delle aziende investe nella comunicazione multicanale per far parlare di sé e per farsi conoscere a nuovi target di riferimento.


\subsubsection{Tipi di promozione}
\begin{itemize}
	\item Promozione verso il consumatore finale, fatte dal produttore o dall'intermediario, mediante campioni gratuiti, coupon, premi, estrazioni ecc.
	\item Promozioni verso gli intermediari. Lo scopo è stimolare costoro	a promuovere a loro volta con efficacia il prodotto. Si tratta di sconti sulle quantità acquistate, campioni gratuiti, premi.
\end{itemize}



\subsection{Le emozioni}
Tutte le emozioni sono, essenzialmente, \textbf{impulsi ad agire}; in altre parole piani d’azione dei quali ci ha dotati l'evoluzione per gestire in tempo reale le emergenze della vita.
Il senso del movimento si rintraccia anche nell’etimo, che ci suggerisce il verbo latino “moveo”, ovvero “muovere”. È nella sua combinazione con il prefisso “e-“ che il lemma genera il suo significato di “muovere da” e ci conduce all’emozione come movimento da, come flusso di un agire che si sposta, che viaggia, che si genera e si sviluppa in un percorso da-a.
L’impulso ad agire viene favorito, a livello cerebrale, dalla presenza di \textbf{marcatori somatici}.
I marcatori somatici hanno la funzione di \textbf{‘registrare’ le nostre esperienze}, contrassegnandole in base alle sensazioni viscerali e non viscerali che proviamo in una data situazione. \textbf{Ci orientano} verso una gamma di alternative possibili di comportamenti, permettendoci di scegliere uno di essi: un marcatore somatico negativo funge da campanello d'allarme per il futuro, inibendo comportamenti futuri, un marcatore somatico positivo
incentiva comportamenti futuri. Ad esempio l’acquisto di un prodotto il cui uso è associato ad emozioni positive oppure il cui posizionamento o la cui comunicazione sono associati a valori che evocano emozioni positive, sono associati ad un marcatore somatico positivo, invece il morso di un cane e il dolore da esso causato è associato ad un marcatore negativo. L’informazione emotiva è trasmessa dai neuroni attraverso una variazione di frequenza dei loro impulsi. La ricerca neurologica ha dimostrato che i neuroni si accendono nello stesso modo e nelle stesse aree sia a seguito di un’azione compiuta dall’individuo sia a seguito di un’azione osservata, come il comportamento osservato nell’ambito di una comunicazione pubblicitaria. I neuroni che si accendono a seguito della semplice osservazione vengono chiamati \textbf{neuroni specchio}. L’esistenza di una forma di rispecchiamento, vale a dire la riproduzione all’interno di noi dello stato in cui si trovano gli altri, è alla base dell’empatia e dell’apprendimento, dell’identificazione e della comprensione delle intenzioni altrui oltre che del desiderio. Il marketing dunque può influire sugli atteggiamenti e sulle motivazioni che determinano il comportamento futuro degli acquirenti-consumatori, innescando, attraverso la comunicazione, un processo di apprendimento passivo, che prescinde dall’esperienza diretta del consumatore.
\paragraph{Il marketing esperienziale} Pone l’accento sulla centralità dell'esperienza d’uso e di consumo del cliente: è quello che un consumatore prova quando entra in contatto con un bene e un servizio e con la comunicazione di questi ultimi. 
Schmitt suddivide le esperienze in 5 tipologie, dette SEM(Moduli Strategici Esperienziali): 
\begin{itemize}
	\item \textbf{feel:} esperienze che suscitano sentimenti ed emozioni
	\item  \textbf{sense:} esperienze legate alla percezione sensoriale, la vista, l’udito, il tatto, il gusto e l’olfatto
	\item  \textbf{think:} esperienze che coinvolgono i processi cognitivi di apprendimento: hanno l’obiettivo di creare esperienze cognitive e di problem-solving
	\item  \textbf{act:} esperienze che spronano il consumatore ad agire, ad assumere determinati comportamenti e stili di vita
	\item  \textbf{relate:} esperienze derivanti da interazioni e relazioni sociali: vengono arricchite le esperienze individuali mettendo in relazione l’individuo con il suo sé ideale, con gli altri individui e con le altre culture. (es. «Think different» di Apple vuole far sentire i suoi clienti parte di un’élite sociale composta da ribelli e creativi)
\end{itemize}
Schmitt propone l’approccio di marketing esperienziale in
contrapposizione all’approccio di marketing classico.
L’autore, in particolare, mette in discussione l’impostazione razionale e utilitaristica tipica del marketing tradizionale che vede il consumatore come un soggetto razionale che decide in base alle caratteristiche e ai benefici funzionali dei prodotti. I consumatori, dice Schmitt, sono sempre più alla ricerca di esperienze che coinvolgano i sensi, il cuore e la mente; essi cercano prodotti, comunicazione e campagne di marketing con i quali relazionarsi e che possano incorporare nel loro stile di vita.

\subsection{Pubblicità}
La pubblicità è una delle componenti dell'azione di
marketing e il suo ruolo è inseparabile da quello degli altri fattori che concorrono alla vendita. In generale la pubblicità può essere efficace solo quando agiscono anche gli altri elementi del piano di marketing: prodotto differenziato, prezzo attraente e distribuzione sufficiente.

La pubblicità risponde ad un bisogno di \textbf{informazione} e si
rivela più utile quando l’acquirente si trova di fronte a prodotti con i quali ha \textbf{scarsa familiarità}, in particolari prodotti le cui caratteristiche non sono evidenti a colpo d’occhio.
Perché una comunicazione pubblicitaria sia veramente
efficace, deve mettere in luce una particolarità specifica, una \textbf{qualità distintiva}, un valore del prodotto che gli conferisca superiorità sui prodotti della concorrenza e che lo \textbf{posizioni nella mente dell'acquirente}. \newline
Obiettivi:
\begin{itemize}
	\item \textbf{Raggiungere un ampio numero} di potenziali compratori target
	\item \textbf{Informare} che un prodotto è disponibile
	\item \textbf{Convincere e persuadere:} orientare il consumatore verso un prodotto, facendolo preferire tra molti altri
	\item \textbf{Ricordare e rinforzare} un acquisto già effettuato
	\item \textbf{Migliorare i rapporti con gli intermediari} poiché offrono prodotti già noti ai consumatori
	\item \textbf{Rafforza l’immagine} di una marca o di un’impresa
\end{itemize}

\subsubsection{Tipi di pubblicità}
\paragraph{Pubblicità di prodotto} ha lo scopo di aumentare le vendite di prodotto rafforzando la
comunicazione del posizionamento scelto. Talvolta l’impresa è in secondo piano.
\paragraph{Pubblicità istituzionale}
ha l’obiettivo di creare un’immagine favorevole all'impresa nel suo complesso. Questo tipo di pubblicità può utilizzare i mezzi tradizionali oppure lo sponsorship: concerti, restauri, avvenimenti
sportivi. L’evento a cui si decide si associare il proprio nome deve essere sempre coerente con il proprio posizionamento.

\subsubsection{Pubblicità e ciclo di vita del prodotto}
Quando la domanda globale è espandibile la pubblicità ha più forte impatto sul mercato, contribuendo in particolare ad accelerare la diffusione del prodotto: la pubblicità svolge un ruolo di catalizzatore della domanda.

Quando il prodotto-mercato è in fase di maturità, la pubblicità ha un ruolo di mantenimento o influisce soprattutto sulle quote di mercato.

\subsubsection{Tecniche di persuasione}
\begin{itemize}
	\item \textbf{Dimostrazione:} è utilizzata per mostrare che cosa fa il prodotto. Più è spettacolare e incredibile, meglio è.
	\item \textbf{Scena di vita:} viene utilizzata per inserire il prodotto al centro di una storia quotidiana, in modo da aumentare il coinvolgimento del fruitore. L’obiettivo è di fare in modo che lo spettatore si identifichi nella situazione a lui familiare. Ha la funzione di: generare un atteggiamento di simpatia, emozione e calore verso il prodotto; mostrare come il prodotto funziona e risolve i problemi.
	\item \textbf{Sottointeso:} basata sull'intesa tra creativi e pubblico; questa tecnica afferma per accenni e rinvii, può servire anche per comunicare dichiarazioni comparative.
	\item \textbf{Ammiccamento:} spesso il riferimento è all’ambito sessuale.
	\item \textbf{Presupposizione:} consiste nel far credere all'ascoltatore che parte del messaggio che sta ricevendo sia scontata, senza bisogno di essere dimostrata.
	\item \textbf{Testimonianza}
	\begin{itemize}
		\item \textbf{Il personaggio famoso:} risulta efficace quando l’immagine del personaggio celebre è in qualche modo correlata al prodotto.
		\item \textbf{L’esperto:} celebre o meno, l’importante è che sia percepito come testimonial credibile.
		\item \textbf{La gente comune:} contrariamente a quanto si può pensare, questa	tipologia ha un’elevata credibilità. Molte interviste realizzate con normali consumatori risultano più efficaci di quelle realizzate con	attori, poiché questi ultimi vengono vissuti per quello che realmente sono, e cioè una finzione.
	\end{itemize}		
\end{itemize}

\subsubsection{Aspetti etici}
Dal punto di vista etico, una frequente critica che viene mossa, è che la pubblicità è in grado di far percepire delle differenziazioni inesistenti nonché consentire di ottenere prodotti con prezzi più elevati rispetto a coloro che non svolgono tale azione.

Dal punto di vista sociale, una frequente critica che viene mossa, è che la pubblicità induce al consumismo, spinge alla soddisfazione materialistica delle esigenze del consumatore ignorando gran parte dei valori dello spirito.

Questa critica va considerata insieme alla manipolazione dei desideri dei consumatori, una pubblicità cosiddetta di tipo subliminale
in cui vengono superate le barriere dell'apprendimento di tipo conscio, agendo direttamente sull'inconscio.

\subsection{Promozione delle vendite}
La promozione delle vendite è un’attività che riunisce un insieme di tecniche e mezzi di comunicazione, messi in atto nell'ambito del piano d’azione commerciale dell'impresa, allo scopo di suscitare nel target prescelto la nascita o l’evoluzione di un comportamento d’acquisto o di consumo a breve o a lungo termine.
\\
A differenza della comunicazione pubblicitaria la promozione opera principalmente sul comportamento del consumatore e meno sulla mente; infatti se gli acquirenti hanno notizia di saldi o vendite promozionali, attivano immediatamente comportamenti d’acquisto. \newline
Le tecniche promozionali vengono così individuate:
\begin{itemize}
	\item \textbf{Riduzioni di prezzo} Si tratta essenzialmente di offrire la stessa cosa ad un prezzo meno elevato, ricorrenti a diversi procedimenti: buono sconto, offerte speciali, 3x2, ritiro di prodotto, ecc
	\item \textbf{Vendita con premi ed omaggi} Agli acquirenti del prodotto vengono offerti gratuitamente piccoli oggetti, immediatamente o successivamente all'acquisto.
	\item \textbf{Prove e campioni} La distribuzione gratuita dei campioni o le degustazioni permettono ai consumatori di provare il prodotto.
	\item \textbf{Giochi e concorsi} Si tratta di gare a carattere ludico, che alimentano a speranza di vincite elevate.
	\item \textbf{Presentazione nel luogo di acquisto} Nei luoghi di vendita il produttore fornisce dimostrazioni circa le caratteristiche del prodotto.
	\item \textbf{Dimostrazioni} Particolarmente efficace per prodotti alimentari e cosmetici.
	\item \textbf{Fiere, mostre e seminari} Particolarmente efficaci per beni strumentali (macchinari) e beni di consumo durevole (mobili).
\end{itemize}

\subsection{Pubbliche relazioni}
Le pubbliche relazioni riuniscono le comunicazioni
elaborate dall'impresa per far conoscere l’esistenza, le azioni e le finalità dell'impresa e sviluppare
un’immagine favorevole nella mente del pubblico in
generale e degli interlocutori istituzionali e commerciali. Nelle pubbliche relazioni, l’obiettivo non è parlare del prodotto, bensì creare e consolidare un atteggiamento positivo nei confronti dell'impresa tra i vari segmenti di pubblico. \newline
Gli strumenti utilizzati dalle pubbliche relazioni possono essere riuniti in quattro categorie:
\begin{itemize}
	\item \textbf{Pubblicazioni}, come riviste aziendali, rapporti annuali, opuscoli per la clientela, cataloghi che al giorno d’oggi sono spesso disponibili sul sito Internet dell'impresa.
	\item \textbf{Eventi o manifestazioni}, quali gare sportive, concerti o	esposizioni sponsorizzate dall’impresa.
	\item \textbf{Informazioni relative all’impresa}, come il lancio di un nuovo prodotto, un contratto importante, un risultato nella R\&S (ricerca e sviluppo). Spetta agli
	esperti di pubbliche relazione organizzare gli incontri con i giornalisti e preparare il comunicato stampa.
	\item \textbf{Mecenatismo} o partecipazione dell'impresa a cause di interesse generale, umanitarie, scientifiche o culturali.
\end{itemize}
Kotler riassume le P.R. nella Parola “PENCILS”: pubblicazioni, eventi,notizie, attività comunitarie, simboli di identità, lobbying e responsabilità sociale.

\subsection{Vendita personale}
Uno degli strumenti di comunicazione di marketing più costoso è costituito dal personale di vendita dell'azienda.
Il personale di vendita ha il vantaggio di essere più efficace rispetto alla pubblicità. Il venditore stabilisce un contatto diretto con il cliente, può condurlo a pranzo, stimolare l’attenzione e l’interesse anche attraverso la fruizione del prodotto. Più il prodotto è complesso e maggiore è la necessità di impiegare personale diretto.

\subsection{Marketing diretto}
Tale forma di marketing, attraverso l’ausilio di uno o più mezzi di comunicazione, si propone di stabilire un dialogo con il mercato a livello di singolo cliente e non di gruppo o massa di clienti.

Gli obiettivi che si vogliono cogliere con l’azione di marketing diretto sono:
\begin{itemize}
	\item creare liste (data base) di potenziali clienti
	\item generare contatti con probabili clienti
	\item indurre al riacquisto i clienti già acquisiti facendo loro pervenire proposte mirate (ad esempio, individuando il nome di coloro che hanno acquistato un pc di recente si può pensare di proporre uno scanner)
\end{itemize}
Il marketing diretto trova oggi una sua espressione attraverso l’uso del social media marketing (Facebook, YouTube, Twitter e LinkedIn).
