\chapter{Ricerca di mercato}
Le ricerche di mercato presentano carattere esplorativo e sono effettuate sia per individuare la “rotta” aziendale, sia per accertare specifici aspetti/bisogni del mercato, come atto propedeutico alla definizione di una strategia o di un piano di fattibilità. La ricerca di mercato è, dunque, l’insieme degli strumenti utilizzati da coloro che producono beni e servizi o che promuovono idee e valori, per mantenersi in contatto con i bisogni e le necessità dei consumatori. \newline
Una ricerca di marketing si compone delle seguenti fasi:
\begin{itemize}
	\item Definizione del problema e sviluppo delle ipotesi
	\item Formulazione del disegno della ricerca
	\begin{itemize}
		\item Costruzione dello o degli strumento/i
		\item Definizione di un piano di campionamento
	\end{itemize}
	\item Racccolta dei dati
	\item Codifica e analisi dei dati
	\item Interpretazione dei risultati
	\item Presentazione finale dei risultati
\end{itemize}
La ricerca è finalizzata ad approfondire caratteristiche e le opportunità del mercato riferimento, nonché a determinare in che misura gli acquirenti differiscono tra loro in termini di bisogni, percezioni e preferenze.

\section{Individuazione della domanda}
Scelta della categoria di potenziali clienti i cui bisogni e desideri si vogliono soddisfare attraverso determinati prodotti e identificazione del numero degli acquirenti, del loro potere d’acquisto e della loro disponibilità all'acquisto. \newline
Per l’analisi delle opportunità ci si può avvalere di due tipologie di dati:
\begin{itemize}
	\item Dati primari già disponibili:
	\begin{itemize}
		\item Fonti statistiche ufficiali (Istat, Eurostat, Ministeri, enti e associazioni per la ricerca economica, associazioni di categoria, Camere di Commercio(CCIAA), banche e imprese, istituti e agenzie per il marketing, fonti informative estere)
		\item Associazioni di categoria
		\item Istituti specializzati (DOXA, Censis, Nielsen)
		\item Letteratura specialistica (riviste e data base bibliografici, atti di conferenze)
		\item Letteratura non specialistica (periodici, testate giornalistiche, pubblicazioni varie, ...)
		\item Rassegna stampa (articoli di quotidiani)
	\end{itemize}
	\item Dati secondari raccolti attraverso la messa a punto di Ricerche di Mercato rivolte ai potenziali consumatori (questionari, interviste, focus group ...)
\end{itemize}

\section{Definizione di un piano di campionamento}
La conoscenza statistica di un fenomeno può essere acquisita sia mediante una rilevazione completa delle sue manifestazioni, sia attraverso una rilevazione parziale che consenta di risalire con buona approssimazione alle caratteristiche complessive del fenomeno. La rappresentatività dei dati raccolti attraverso rilevazioni complete non è sempre superiore a quella riscontrabile utilizzando metodologie campionarie, dato che l’errore di osservazione nelle prime è spesso altrettanto consistente che nelle rilevazioni parziali.
\begin{description}
	\item[Popolazione:] qualsiasi insieme di elementi simili tra loro per una o più caratteristiche, che rappresentano l’oggetto di studio di una particolare indagine.
	\item[Campione:] si definisce quella parte limitata di popolazione che viene presa in esame.
	\item[Campionamento:] rappresenta il procedimento di individuazione del campione, cioè di un insieme di osservazioni rappresentativo della popolazione.
	\item[Parametro:] si intende la misura di una caratteristica della popolazione.
	\item[Statistica:] si intende la misura corrispondente nel campione.
\end{description}
Le indagini campionarie sono preferibili a quelle complete nei seguenti casi: 
\begin{itemize}
	\item quando la popolazione è di \textbf{dimensione} infinita o è un’entità astratta
	\item quando le rilevazioni di tutte le unità della popolazione da analizzare comportano \textbf{costi elevati.}
	\item quando i risultati della ricerca devono essere forniti in \textbf{tempi brevi.}
	\item quando la rilevazione, misurazione e controllo delle informazioni comporta la \textbf{distruzione delle unità esaminate}, come nel controllo statistico di qualità della produzione.
	\item nelle prove di mercato di un prodotto, quando esse non si possono svolgere sull'intero mercato, per motivi di \textbf{riservatezza verso la concorrenza} o per incompleta definizione di alcune variabili di marketing.
\end{itemize}

\subsubsection{Tipologie di errori di campionamento}
\paragraph{Errore casuale:} esso risulta determinato dalle fluttuazioni casuali del campione e può essere stimato ed eventualmente ridotto aumentando la numerosità del campione stesso.

\paragraph{Errore sistematico:} esso risulta più difficile da cogliere poiché direttamente connesso alle problematiche relative al piano di campionamento e quindi al metodo di rilevamento utilizzato e alle tecniche impiegate.

\subsection{Tecniche di campionamento}
Le tecniche di campionamento possono essere suddivise in due principali categorie:
\begin{itemize}
	\item quelle che danno luogo a campioni probabilistici
	\item quelle che producono campioni non probabilistici
\end{itemize}
\paragraph{Campionamento probabilistico}
Si ha un campione probabilistico quando ad ogni unità della popolazione compete una probabilità nota e non nulla di essere estratta. \newline
I tipi più comuni di campionamento probabilistico sono i seguenti:
\begin{itemize}
	\item campionamento casuale semplice
	\item campionamento stratificato
	\item campionamento per aree o per grappoli
\end{itemize}

\subsubsection{Campionamento casuale semplice}
Le unità vengono estratte singolarmente mediante un processo di scelta casuale, in modo che ogni unità abbia la medesima probabilità di venire selezionata e che la sua estrazione sia indipendente da quella delle altre. \newline
È necessario poter disporre di una lista completa e precisa delle unità di campionamento da cui estrarre i singoli elementi. Si possono adottare procedure diverse per ottenere il campione, ma esse comunque devono garantire l’equiprobabilità di estrazione ad ogni unità della lista. \newline
Un’ampia numerosità campionaria può condurre ad un campione casuale più rappresentativo.

\paragraph{Vantaggi} 
Il campionamento casuale semplice risulta il metodo migliore quando si conosce poco o nulla riguardo alla popolazione in esame. \newline
Con una numerosità elevata, il campione casuale diviene tanto più rappresentativo quanto più le caratteristiche essenziali dei dati estratti tendono a rispecchiare quelle della popolazione. \newline
La responsabilità del ricercatore è limitata, in quanto egli è tenuto a compiere soltanto alcune scelte preliminari, dal momento che la rappresentatività del campione scaturisce in maniera pressoché automatica dall'estrazione casuale.

\paragraph{Svantaggi} 
Tale campionamento richiede di poter disporre di una lista completa e corretta degli elementi che costituiscono la popolazione. \newline
Vi sono inevitabili difficoltà relativamente alla fase di contatto con gli individui estratti, il cui reperimento spesso costituisce un onere assai gravoso. \newline
Questo tipo di campionamento espone all'errore casuale, che può tuttavia essere contenuto aumentando la numerosità campionaria, determinando tuttavia un incremento dei costi complessivi della ricerca. \newline
Ritroviamo due tipologie di campionamento casuale semplice: senza e con ripetizione. \newline
Nella prima tipologia ogni unità campionaria via via estratta viene reinserita nella popolazione di cui fa parte; attraverso tale modalità le probabilità di estrazione non vengono modificate dalle successive estrazioni, dando luogo a stime più attendibili.\newline
Quando si dispone di grandi popolazioni, si assume che l’estrazione mediante la tecnica del campionamento casuale senza reinserimento non modifichi la probabilità di selezione degli altri elementi. \newline
Nelle piccole popolazioni, invece, per mantenere la casualità nell'estrazione campionaria, è preferibile porre ciascuna unità nella condizione di essere nuovamente selezionata.

\subsubsection{Campionamento stratificato}
Il campionamento stratificato viene utilizzato quando nella popolazione di riferimento sono possibili da individuare naturalmente delle sottopopolazioni o degli strati omogenei. \newline
L’universo dell'indagine viene considerato suddiviso in sub-universi esprimenti ciascuno le variabili di rilievo per l’indagine e da ogni strato si estraggono con criteri di causalità, le unità campionarie. I gruppi di unità si presentano omogenei al loro interno – rispetto ai caratteri considerati – e quanto più possibile eterogenei fra di loro.

\paragraph{Vantaggi} 
Da un campione stratificato si ottengono stime più attendibili rispetto ad un campione casuale semplice della stessa dimensione. Si ottengono, inoltre, delle stime più attendibili riducendo la numerosità campionaria. \newline
La stratificazione comporta, inoltre, una certa convenienza gestionale ed organizzativa, poiché il ricercatore può ottimizzare la realizzazione delle rilevazioni e ridurre i tempi e i costi riferendosi a variabili naturali facilmente identificabili.

\paragraph{Svantaggi}
Il campione stratificato può essere utilizzato solo nel caso in cui abbiamo informazioni sufficienti riguardo alle distribuzioni e alle proporzioni delle differenti tipologie nella popolazione che andranno a costituire i diversi strati.

\paragraph{Campionamento non probabilistico}
Nel campionamento non probabilistico le singole unità della popolazione non hanno la stessa probabilità di entrare a far parte del campione. \newline
Tra i campioni non probabilistici ritroviamo in particolare:
\begin{itemize}
	\item campionamento non probabilistico per quote
	\item campionamento non probabilistico a valanga
	\item campionamento non probabilistico accidentale
\end{itemize}

\subsubsection{Campionamento non probabilistico per quote}
\begin{enumerate}
	\item La popolazione di riferimento viene suddivisa in classi o sottogruppi omogenei secondo, a titolo esemplificativo, il genere, l’età e il titolo di studio.
	\item Dai dati dell'ultimo censimento si ricava il peso percentuale di ogni classe.
	\item Il totale delle unità campionarie viene successivamente suddiviso tra le classi in modo da rispecchiare le proporzioni esistenti nella popolazione.
	\item Si perviene, dunque, alla definizione delle quote, cioè del numero delle interviste da effettuare in ciascuna classe.
\end{enumerate}

\paragraph{Vantaggi} 
\begin{itemize}
	\item Il campionamento per quote permette di contenere i tempi e i costi della ricerca.
	\item Consente ai rilevatori di non spostarsi eccessivamente, poiché è sufficiente sostituire opportunamente un soggetto non raggiunto invece di ricontattarlo.
	\item Si tratta di una procedura particolarmente utile quando vengono richiesti risultati urgenti.
	\item Essendo questo metodo molto più flessibile di quello casuale, risulta particolarmente interessante e adeguato per chi si occupa di marketing.
\end{itemize}

\paragraph{Limiti} 
\begin{itemize}
	\item Poiché il metodo non comporta il controllo dei rifiuti a collaborare, è possibile che vi sia un minore impegno da parte dei rilevatori per ottenere l’intervista.
	\item Esiste il rischio di distorsione per quanto attiene l’inclusione di coloro che sono difficilmente reperibili.
	\item Esiste la possibilità di una sottostima della variabilità, se il rilevatore tende ad avvicinare persone tra loro simili (errore sistematico).
	\item E’ possibile una iniziale distorsione dei dati, quando le quote vengono stabilite su statistiche non aggiornate.
\end{itemize}

\subsubsection{Campionamento non probabilistico a valanga}
Il campione a valanga viene impiegato nelle indagini sulle popolazioni rare, e consiste nello scegliere un gruppo iniziale di persone, dalle quali poi ottenere nomi e indirizzi di altre unità appartenenti alla stessa popolazione. È utile ad esempio, nel caso di gruppi linguistici o etnici di piccole dimensioni o dispersi sul territorio.
\begin{description}
	\item[Prima fase] si individuano e si intervistano alcune persone in possesso delle caratteristiche che si vogliono analizzare. Questi soggetti vengono considerati informatori utili ad individuare altre persone idonee ad essere incluse nel campione.
	\item[Seconda fase] si intervistano queste persone che, a loro volta, ci permettono di contattare altri soggetti ancora da intervistare nella terza fase, e così via.
\end{description}

\subsubsection{Campionamento non probabilistico accidentale}
Il campionamento accidentale o di convenienza è la tecnica più rudimentale di selezione, in quanto il ricercatore si limita a scegliere come rispondenti le prime persone che capitano. I soggetti vengono acquisiti in modo indiscriminato fino a quando non si raggiunge l’ampiezza prestabilita dal campione. Con tale tecnica si ottiene un risparmio considerevole di tempo e di denaro, ma essa è particolarmente soggetta ad errori sistematici, quindi viene utilizzata quasi esclusivamente in indagini esplorative.

\section{Tipi di ricerca di mercato}
Le ricerche di mercato si distinguono in:
\begin{itemize}
	\item ricerche esplorative
	\item ricerche descrittive o correlazionali
	\item ricerche causali o esplicative
\end{itemize}

\subsection{Ricerche esplorative}
Sono volte a chiarire la natura di un problema e a tradurlo in specifiche ipotesi di ricerca. Come metodi si avvale di ricerca documentale e di studi e strumenti qualitativi.

\subsection{Ricerche descrittive o correlazionali}
Presuppongono una buona definizione del problema e sono volte a fornire il maggior numero di informazioni per descrivere un problema senza però stabilire rapporti di causa-effetto tra attori o di formulare previsioni. \newline
Come metodi si avvale di interviste strutturate faccia a faccia o telefoniche e di questionari.

\subsection{Ricerche causali o esplicative}
Sono volte a identificare la relazione di causa-effetto tra una o più variabili e fornire una spiegazione ai fenomeni descritti in modo da formulare previsioni per il futuro. Implicano l’esistenza di ipotesi ben definite e il ricorso a una metodologia precisa e permettono la creazione di modelli in grado di simulare strategie di marketing. Come metodi si avvale di esperimenti di laboratorio e sul campo.

\section{Tecniche di indagine}
\begin{itemize}
	\item \textbf{Qualitative:} colloqui, interviste in profondità individuali e/o collettive, focus group, tecniche proiettive (descrizione di terze persone, completamento di frasi, libere associazioni di parole, interpretazione di disegni)
	\item \textbf{Quantitative:} questionario, interviste semi-strutturate, interviste strutturate.
\end{itemize}

L’informazione qualitativa sorregge processi decisionali di due tipi: situazioni caratterizzate da un’elevata indeterminatezza e problemi ancora non chiaramente definiti, e situazioni caratterizzate da un’elevata soggettività e sottoposti a valutazioni di natura verbale. L’informazione qualitativa produce per sua natura delle indicazioni. \newline
Le indicazioni portano all'attenzione del ricercatore alcuni aspetti del problema di marketing da lui non direttamente percepiti. \newline
Riesce inoltre a cogliere la la subliminarietà: cioè la capacità dell'informazione qualitativa di andare oltre il semplice dichiarato dell'intervistato, cogliendone aspetti motivazionali, di desiderio, di aspettative. \newline
L’informazione quantitativa sorregge elementi di marketing per loro natura quantitativi e soggetti a valutazioni di natura numerica. Ciò permette la comparabilità: è possibile monitorare nel tempo un fenomeno di marketing, istituire relazioni incrociate fra misure diverse ed effettuare controlli di performance rispetto a standard di riferimento.

\paragraph{Approccio integrato}
Individuazione delle ipotesi e messa a punto dello strumento di rilevazione, avvalendosi del contributo informativo prodotto dall'indagine qualitativa; e indagine quantitativa estensiva, tesa a misurare le dimensioni del fenomeno emerse dalla fase qualitativa.

\section{Questionario}
Il questionario è uno strumento di rilevazione dei dati, ma si può anche chiamare tecnica di raccolta dei dati. \newline
Un questionario deve svilupparsi in modo logico e tale da mantenere alto l’interesse dell'intervistato. \newline
Dalla letteratura è possibile trarre alcuni suggerimenti:
\begin{itemize}
	\item raggruppare gli item che si riferiscono alla stessa area tematica ed esaurirla prima di passare alla successiva
	\item passare dalle questioni generali a quelle più specifiche
	\item porre le domande complesse in una posizione intermedia
	\item se è necessario usare una sequenza temporale, è bene porre prima  le domande riguardanti il passato, poi quelle relative al presente, quindi quelle relative al futuro
	\item affrontare prima i temi più concreti o più conosciuti, quindi quelli più astratti o meno conosciuti
	\item all’inizio del questionario dovrebbero essere poste domande a cui è facile rispondere, non dovrebbero essere percepite come minacciose né richiedere un eccessivo sforzo di memoria o di concentrazione e se possibile essere interessanti; porre alla fine del questionario le domande che possono suscitare ostilità e resistenze
	\item separare le domande che fungono da controllo dell'attendibilità delle risposte.
\end{itemize}

\subsection{Pretest}
È condotto su un numero limitato di soggetti (12-25), ma è indispensabile che questi abbiano caratteristiche simili a quelle della popolazione che si vuole studiare.
Lo scopo è miglioramento dello strumento prima che esso venga effettivamente impiegato, ciò avviene tramite l'individuazione e la correzione delle principali e debolezze tra cui:
\begin{itemize}
	\item gli errori di interpretazione
	\item le domande superflue
	\item le domande mancanti
	\item le domande inappropriate, ridondanti o confuse
\end{itemize}

\subsection{Presentazione}
La presentazione del questionario ha almeno tre funzioni:
\begin{enumerate}
	\item presentare le persone e/o l’organizzazione che la conducono
	\item indicare gli obiettivi dell’indagine
	\item sottolineare il valore della collaborazione dell'intervistato
\end{enumerate}
Le istruzioni devono comunicare con chiarezza le regole da seguire nella compilazione. Per assicurarsi una regolare compilazione è necessario riportare per iscritto anche alcuni esempi, utili soprattutto per la scelta fra le alternative di risposta. È importante la garanzia dell'anonimato, che si esprime nel cosiddetto “segreto statistico”: ciò assicura che le informazioni ricevute dagli intervistati saranno rese note solo ed esclusivamente sotto forma di dati globali.

\subsection{Titolo}
E’ necessaria molta cautela nella scelta del titolo del questionario.\newline
Un titolo può essere utile anche per incentivare la collaborazione dei soggetti. \newline
Un titolo che cita esplicitamente gli obiettivi della ricerca può essere inadeguato. Vale il suggerimento di sceglierlo dopo un’attenta rilettura del questionario ed una estrapolazione dei suoi contenuti principali.

\subsection{Linguaggio}
Deve essere uno strumento comprensibile per tutti quando deve essere somministrato ad un campione eterogeneo di soggetti.\newline
È buona regola evitare i termini tecnici o specialistici. L’uso di una terminologia inadeguata può provocare difficoltà nella comunicazione o addirittura la sua interruzione, una caduta della motivazione, imbarazzo ad ammettere di non conoscere ciò di cui si sta parlando e la tendenza a rispondere a caso pur di non chiedere un chiarimento. Anche l’uso di alcuni concetti come, ad esempio, le percentuali e le proporzioni possono risultare di difficile comprensione.\newline
È un errore usare espressioni colloquiali, dialettali o di gergo. È consigliabile usare una forma impersonale di abituale cortesia.

\subsection{Lunghezza}
Limitare il più possibile la lunghezza delle domande. Un questionario lungo consente di raccogliere maggiori informazioni e quindi di analizzare e comprendere meglio i diversi aspetti del problema oggetto di indagine. Nel contempo, però, un questionario lungo può determinare una progressiva stanchezza e riduzione della motivazione del rispondente. \newline
Una eccessiva lunghezza potrebbe provocare eventuali rifiuti a continuare o un numero più alto di risposte date a caso, a causa
di un abbassamento della motivazione del soggetto a collaborare. La somministrazione di un questionario non dovrebbe superare in ogni caso i 60 minuti. \newline
Può essere consigliabile in tal caso una pausa in corrispondenza con un cambiamento di area tematica, dopo i primi 20 minuti.

\subsection{Domande chiuse}
Domande chiuse o ad alternative fisse o strutturate: esse sono accompagnate da una lista di alternative (o categorie o modalità di risposta) fra le quali il soggetto deve scegliere quella o quelle che meglio rappresentano la sua risposta o il suo giudizio. \newline
Esse possono essere distinte in base al formato di risposta in:
\begin{itemize}
	\item domande a risposta alternativa o a scelta forzata: il soggetto può rispondere con un “sì” o con un “no”, oppure scegliendo fra due alternative opposte.
	\item domande a risposta graduata: il soggetto può scegliere tra varie alternative presentate in un certo ordine.
	\item domande con lista di preferenza o a scelta multipla: il soggetto nel rispondere deve compiere una scelta da un elenco di possibili risposte e tale scelta può cadere su una o più alternative di risposta.
\end{itemize}

\subsubsection{Vantaggi}
Sono più facili e veloci da somministrare e codificare. La lista delle categorie di risposta che accompagna una domanda chiusa può aiutare l’intervistato a comprendere meglio la domanda e a focalizzarsi proprio su quegli aspetti che interessano al ricercatore. L’uniformità della struttura di riferimento rappresentata dalle alternative di risposta facilita il confronto tra soggetti e le analisi statistiche. È più facile ottenere risposte su argomenti delicati.

\subsubsection{Svantaggi}
Le categorie di risposta predeterminate possono “suggerire” una risposta. \newline
La definizione della lista di alternative si basa su un’assunzione di coincidenza tra gli schemi del ricercatore e quelli di tutti i soggetti, tuttavia ogni intervistato può dare a una domanda un’interpretazione personale che non traspare dalle sue risposte. Alcune persone, le più interessate e motivate, possono reagire negativamente alle domande chiuse, in particolare se ritengono che nessuna delle categorieproposte può esprimere l’opinione che hanno elaborato.

\subsection{Domande aperte}
Domande aperte: le possibilità di risposta non sono stabilite a priori e non suggerendo alcuna alternativa di risposta consentono al soggetto di rispondere in base alla propria struttura di riferimento e di rilevare ciò che è più rilevante per il soggetto stesso.
\subsubsection{Vantaggi}
Permettono ai soggetti di rispondere in base alla propria struttura di riferimento senza essere influenzati dalle alternative di risposta proposte dal ricercatore. \newline
Sono utili quando non è possibile prevedere tutta la lista delle categorie di risposta o quando questa è troppo numerosa. \newline
Secondo alcuni autori, sono meno sensibili agli effetti della desiderabilità sociale delle risposte.
\subsubsection{Svantaggi}
Gli intervistati sono spesso più restii ad esprimere la loro opinione. Sono più difficili da codificare e analizzare. Richiedono un compito di recupero delle informazioni più gravoso cognitivamente e potenzialmente soggetto a varie distorsioni.

\subsection{Scale di giudizio}
Esse cercano di rilevare gli atteggiamenti e le valutazioni del soggetto chiedendogli di esprimerli nei termini di un giudizio categoriale o numerico. \newline
Si distinguono in:
\begin{itemize}
	\item scale tipo Likert
	\item scale a somma 100
	\item scale a 100 punti
	\item differenziale semantico
\end{itemize}

\subsubsection{Scale tipo Likert}
Esse sono costituite da categorie di risposta ordinate, di numeri, di linee o di categorie verbali. Il cui numero è variabile, anche se generalmente sono 5 o 7. Ai soggetti si chiede di scegliere una sola alternativa. Le espressioni verbali possono essere le seguenti: assolutamente d’accordo, abbastanza d’accordo, poco d’accordo, né d’accordo né in disaccordo, poco in disaccordo, abbastanza in disaccordo, assolutamente in disaccordo.

\subsubsection{Scale a somma 100}
Esse prevedono che il soggetto giudichi le alternative proposte avvalendosi di un criterio di valutazione basato sulla distribuzione di un totale di 100 punti tra le alternative stesse. Il soggetto deve indicare le percentuali relative a ciascuna alternativa in modo che la somma delle stesse risulti pari a 100. Alla base della procedura si colloca l’assunto che il giudizio da parte dei soggetti avvenga previa considerazione d’insieme di tutte le alternative di risposta.\newline
Il totale dei punti attribuiti ad ogni alternativa è infatti il risultato di molteplici confronti messi in atto al fine di stabilire la relazione esistente fra ciascuna alternativa e tutte le altre.\newline
Uno dei limiti applicativi di questo tipo di scala si evidenzia nel caso di un numero piuttosto elevato di alternative di risposta da giudicare. È infatti ipotizzabile la difficoltà incontrata dal soggetto nel fornire una valutazione basata sul confronto contemporaneo di tutte le alternative. Per ovviare a tale limite si possono raggruppare più alternative in un singolo item.

\subsubsection{Scala a 100 punti}
Esse offrono la possibilità al soggetto di attribuire ad ogni alternativa di risposta 100 punti. Tale procedura permette di assegnare un valore numerico da 1 a 100 ad ogni categoria di risposta, sostituendo così il metodo tradizionale che prevede una classificazione delle alternative per ordine di importanza. Il vantaggio di questa scala deriva dal fatto che il soggetto non viene obbligato a fare una scelta tra le alternative di risposta ma è libero di attribuire a ciascuna di esse il valore che ritiene più opportuno.

\subsubsection{Differenziale semantico}
Il soggetto deve valutare un concetto scegliendo uno dei sette punti interposti tra due polarità, definite da aggettivi di significato opposto. Il soggetto ha il compito di comunicare le sue prime impressioni senza “pensare troppo a qual è la risposta giusta: lo scopo è quello di scoprire quali sono i sentimenti che le persone provano nei confronti di certi argomenti, favorendo la formulazione di una risposta sincera.

\subsection{Domande filtro}
Esse mirano a rilevare se a un soggetto vanno rivolte le domande successive del questionario. Prima di affrontare un argomento con l’intervistato può essere utile infatti sapere se egli possiede su questo un certo grado di competenza o di esperienza. Una domanda condizionata segue una domanda filtro. Operando una selezione dei soggetti, consentono un notevole risparmio di tempo, tuttavia si avrà una diminuzione della numerosità del campione, del quale bisognerà controllare di volta in volta la rappresentatività.

\subsection{Mettere a proprio agio l'intervistato}
\subsubsection{domande imbarazzanti/intrusive/delicate}
È importante non mettere in imbarazzo l’intervistato, soprattutto quando si affrontano argomenti delicati oppure quando le persone possono non conoscere bene l’argomento.

\textit{Formulazione sbagliata:} Hai mai fatto uso di droghe?
\textit{Formulazione corretta:} Molte persone hanno fatto uso droghe in alcuni momenti della loro vita, a lei è successo qualche volta? \newline
Quando si affrontano temi delicati è importante presentare l'argomento come una situazione già affrontata da altri su cui non si può rispondere senza riflettere; una risposta eventualmente positiva non farebbe sentire a disagio chi risponde perché gli è appena stato ricordato, nella domanda stessa, che gli altri si sono trovati nella sua stessa situazione.

\subsubsection{Domande di argomento sconosciuto}

\textit{Formulazione sbagliata:} Secondo voi quale corrente artistica sta maggiormente influenzando i giovani musicisti italiani? 

\textit{Formulazione corretta:} Avete qualche idea su quale corrente artistica stia maggiormente influenzando i giovani musicisti italiani? \newline
Formulare quindi la domanda in modo da mettere a proprio agio l'intervistato anche nel caso di una risposta negativa che denota ignoranza sull'argomento.

\subsection{Domande da evitare}
\subsubsection{Domande con negazione (peggio se multipla)}
Per quanto possibile, vanno evitate formulazioni di domande o affermazioni in cui si fa uso della negazione o della doppia negazione.
\textit{Formulazione sbagliata:} Penso che gli immigrati extracomunitari NON meritino la fiducia degli italiani.
\textit{Formulazione corretta:}
Penso che gli immigrati extracomunitari meritino la fiducia degli italiani. \newline
Nel dare la risposta un intervistato farà prevalere la reazione al termine, immigrati, invece di attenersi al significato linguistico della frase, se xenofoba potrà rispondere "completamente contrario", sottintendendo "agli immigrati", equivocando sul significato della domanda.

\subsubsection{Domande doppie}
Sono doppie tutte le domande in cui una parte della risposta potrebbe positiva e l’altra negativa.
\paragraph{Formulazione sbagliata:} Si ritiene soddisfatto delle mansioni svolte e della posizione occupata nel suo attuale lavoro?
\paragraph{Formulazione corretta:} Si ritiene soddisfatto delle mansioni svolte nel suo attuale lavoro? E della posizione occupata?

\subsubsection{Domande ambigue}
È importante che le domande siano formulate in modo da contenere informazioni sufficienti a non risultare ambigue. Se si vuole che i gli intervistati rispondano tutti alla medesima domanda bisogna evitare che gli intervistatori siano costretti ad aggiungere “parole proprie” per specificare una domanda incompleta.
\paragraph{Formulazione sbagliata:} La mattina consuma una colazione? \newline
\begin{itemize}
	\item Non chiarisce da cosa sia costituita una colazione
	\item Non è chiaro fino a che ora del mattino un pasto possa essere considerato una colazione
	\item Non è chiaro se la domanda si riferisce ad un consumo abituale o a un giorno preciso
\end{itemize}

\paragraph{Formulazione corretta:} Per i nostri scopi consideri colazione un pasto costituito almeno da una bevanda (te, latte, caffè, ...) e un alimento come brioches, cereali, biscotti, toast o frutta, consumato prima delle 10 del mattino. Secondo questa definizione negli scorsi 7 giorni quante volte ha consumato una colazione?

\paragraph{Formulazione sbagliata:} Quanto è importante che un telefonino abbia la fotocamera?

\paragraph{Formulazione corretta:} Quanto è importante \textit{per te} che un telefonino abbia la fotocamera?
Nella prima formulazione della domanda dell'esempio non c'è una chiara indicazione del criterio da considerare e alcuni intervistati potranno dare una risposta basandosi sulla loro opinione, mentre altri valuteranno le caratteristiche di un telefonino facendo riferimento genericamente ai possibili acquirenti.

\subsubsection{Domande pilotanti}
E’ il caso in cui il testo della domanda indirizza l’intervistato, più o meno apertamente, verso una alternativa di risposta.
\paragraph{Formulazione sbagliata:} Quanto è \textit{positivo} il suo giudizio su Matteo Renzi?

\paragraph{Formulazione corretta:} Qual è il suo giudizio su Matteo Renzi?

\subsubsection{Domande oggettive} 
Bisogna preferire domande oggettive
\paragraph{Domanda non oggettiva:} Studi il pomeriggio?
\paragraph{Domanda oggettiva:} Quante ore hai studiato oggi pomeriggio?
\paragraph{Domanda non oggettiva:} Ti piace andare a teatro?
\paragraph{Domanda oggettiva:} Quante volte sei andato a teatro nell'ultimo mese?
