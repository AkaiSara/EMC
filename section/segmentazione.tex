\chapter{Segmentazione, targeting e posizionamento}
\section{Segmentazione}
La segmentazione è un processo mediante il quale il mercato viene suddiviso in un numero limitato di segmenti, ovvero di gruppi di consumatori, sufficientemente omogenei come motivazioni e comportamenti al loro interno e, invece, sufficientemente eterogenei fra di loro, così da aspirare a diversi marketing mix.(Collesei, 1994). \newline
La segmentazione richiede una dose di creatività, e la base scelta per segmentare il mercato è altrettanto importante quanto il maggiore o minore grado di che si ritiene di accettare all’interno dei segmenti (Abell \& Hammond, 1986). \newline
Nella definizione della propria strategia di segmentazione le imprese debbono valutare attentamente l’attrattiva dei segmenti e decidere in quale segmento posizionarsi. Nel fare ciò, devono fare molta attenzione a non essere attratte dalla cosiddetta miopia di marketing . I segmenti di maggiore consistenza, a motivo delle maggiori vendite potenziali che sembrano offrire, sono quelli che attraggono di più l’interesse delle imprese, specie di quelle di grandi dimensioni. Spesso tuttavia in questi segmenti si verifica un’elevata concentrazione di concorrenti. Risulta perciò più difficile per l’impresa ottenere una adeguata quota di mercato. Al contrario, la scarsa presenza di concorrenti, in particolare di dimensioni elevate, rende più congeniale alle piccole imprese la scelta di segmenti minori. Tale strategie viene definita di nicchia. \newline
Una volta effettuata la segmentazione, e quindi individuata una suddivisione del mercato in segmenti, si presentano all’impresa due principali opportunità:
\begin{itemize}
	\item \textbf{segmentazione concentrata}: scegliere un unico segmento in funzione del quale creare un unico prodotto;
	\item \textbf{segmentazione multipla}: scegliere più segmenti per ciascuno dei quali proporre differenti prodotti.
\end{itemize}

\subsection{Vantaggi della segmentazione}
\begin{itemize}
	\item Consente di individuare e confrontare con maggiore chiarezza le opportunità di mercato.
	\item Migliora la conoscenza sul comportamento del compratore.
	\item Permette di studiare i prodotti e servizi in base alle specifiche esigenze dei compratori.
	\item Promuove la sensibilità dell’impresa a percepire i mutamenti che si verificano nella domanda.
	\item Facilita l’analisi del mercato e della concorrenza, nonché la scelta del posizionamento.
	\item Permette di aumentare la fedeltà dei clienti alla marca.
\end{itemize}

\subsection{Svantaggi della segmentazione}
\begin{itemize}
	\item Per ciascun segmento di mercato è necessario sostenere distinti costi di ricerca, sviluppo e lancio di prodotti. Il costo totale che ne risulta è ovviamente superiore a quello che si avrebbe se il prodotto fosse uno solo.
	\item In particolare l’impresa può sostenere costi più elevati di ricerche di mercato poiché le conoscenze del comportamento dei potenziali compratori debbono essere approfondite e ciò per più segmenti.
\end{itemize}

\section{Posizionamento}
Il posizionamento può essere definito come l’insieme di iniziative volte a definire le caratteristiche del prodotto e ad impostare il marketing mix più adatto per attribuire una certa posizione al prodotto nella mente del consumatore (Kotler, 1984). \newline
Il posizionamento è una tecnica di marketing finalizzata ad associare un’idea/valore positiva/o ad un prodotto, nella mente del consumatore. \newline
{\Large Esempi:}
\begin{itemize}
	\item Mercedes: stile ed eleganza
	\item BMW: prestazioni elevate
	\item Volvo: sicurezza
	\item Fiat: qualità/prezzo
\end{itemize}
	
\subsection{Processo di posizionamento}
\begin{itemize}
	\item Individuazione delle richieste del consumatore.
	\item Individuazione dei concorrenti.
	\item Determinazione delle posizioni dei concorrenti.
	\item Scelta della posizione.
	\item Verifica degli effetti della scelta:
	\begin{itemize}
		\item In funzione delle preferenze dei clienti
		\item In relazione all’ampiezza della popolazione obiettivo
		\item In riferimento alle risorse dell’impresa
		\item In relazione agli intermediari
	\end{itemize}
	\item Realizzazione e controllo della posizione.
\end{itemize}

\subsection{Strategie di posizionamento}
\begin{itemize}
	\item \textbf{Posizionamento di attributo.} L’impresa si posiziona su un determinato attributo o caratteristica, senza esplicitamente affermare nessun beneficio (es. l’università più antica; albergo più panoramico).
	\item \textbf{Posizionamento di vantaggio o beneficio.} Il prodotto promette un vantaggio (Volvo - più sicura; AZ Complete White -denti più bianchi).
	\item \textbf{Posizionamento di impiego/applicazione.} Il prodotto viene presentato come il migliore in un certo campo di applicazione (es. Nike per le scarpe da corsa; Google per i motori di ricerca).
	\item \textbf{Posizionamento di categoria merceologica.} L’impresa punta ad essere il leader di un determinato settore merceologico (Post-it nel settore dei foglietti colorati semi adesivi).
	\item \textbf{Posizionamento competitivo.} Viene affermato che il prodotto è superiore o differente rispetto ad un analogo prodotto della concorrenza.
\end{itemize}

\subsection{Il posizionamento del valore}
È necessario inoltre che l’impresa definisca la posizione dei propri prodotti in riferimento ai relativi prezzi, in funzione del fatto che i clienti associano ad un determinato prezzo un certo valore del prodotto. \newline
Kotler propone cinque strategie:
\begin{enumerate}
	\item di più a più (prodotti ad alta qualità e prezzo elevato es. Chanel, Ferrari, ...)
	\item di più per lo stesso (prodotti ad alta qualità e prezzo medio es. Lexus)
	\item lo stesso a meno (prodotti a media qualità e prezzo basso, Acqua Guizza, alcuni prodotto di marca all’interno dei Discount)
	\item meno per molto meno (prodotti a bassa «qualità» e prezzo basso, es. Raynair, “prodotti a un Euro”)
	\item di più a meno (prodotti ad alta qualità e prezzo
	basso, es. negozi outlet)
\end{enumerate}

\subsection{Riposizionamento}
Il riposizionamento o rebranding di un marchio è quel processo attraverso il quale un prodotto, anche attraverso prodotti correlati, viene reimmesso nel mercato con una diversa identità. \newline
Riposizionare l’immagine di un’azienda e dei suoi prodotti è un’operazione di comunicazione e marketing molto delicata e complessa. Questo tipo di operazione si porta a termine per vari motivi tra cui: acquisire nuove fasce di mercato, adeguamento all'evoluzione del mercato, far fronte a un nuovo concorrente, per favorire un ciclo di vita più lungo del prodotto/servizio ecc. \newline
Si può parlare di:
\begin{itemize}
\item \textbf{rebranding totale} : si verifica un cambiamento radicale del logo, dell'immagine, delle strategie di marketing e di vendita, delle politiche pubblicitarie;
\item \textbf{rebranding parziale} : si verificano piccole modifiche volte a migliorare la percezione del marchio da parte dei consumatori.
\end{itemize}

\subsubsection{Co-branding}
American Marketing Association (AMA) definisce il brand come “un nome, un termine, un segno, un simbolo, un disegno o una loro combinazione che identifica un prodotto o servizio di un venditore e che lo differenzia da quello del concorrente”. Il co-branding può essere definito come una forma di cooperazione tra due marchi, avente il fine di trasferire specifiche valenze positive da una marca all’altra – dalla marca invitata che gode una specifica notorietà, alla marca ospitante – ed ottenere un prodotto di maggiore qualità. Si ha co-branding (condivisione di marchi) quando un prodotto è contrassegnato da marchi che fanno riferimento a proprietari differenti. \newline
Tipologie:
\begin{itemize}
	\item \textbf{Funzionale:} la marca ospitante si accorda con un partner avente specifiche caratteristiche distintive/elementi tangibili; il consumatore percepirà il prodotto caratterizzato da una qualità superiore rispetto alla restante offerta del mercato, in quanto le qualità positive del secondo brand si trasferiscono sull’altro e viceversa.
	\item \textbf{Simbolico:} consiste nell’associare alla marca del produttore una seconda marca, caratterizzata da attributi simbolici di tipo psico-sociale o esperienziali addizionali. Consente, per una durata generalmente breve, di mirare a quel segmento di clientela potenziale che presenta una consonanza particolare con la marca invitata.
\end{itemize}
Vantaggi:
\begin{itemize}
	\item Incremento del valore percepito dell’offerta, con conseguente possibilità di	aumento del prezzo. L’assemblaggio di una tecnologia nota consente un aumento notevole del prezzo di un prodotto di marchio non noto (Es. Fiat e Chrysler).
	\item Arricchimento della gamma, l’arricchimento della gamma delle soluzioni offerte al consumatore consente di aumentare anche la soddisfazione complessiva (Es. Fiat e Gucci).
	\item Incrementare le vendite, spesso acquisendo quote di clienti del marchio partner. Coca Cola grazie all’accordo con Nutrasweet, azienda leader nel settore di prodotti dietetici, ha introdotto la Diet Coke.
	\item Capillarità della distribuzione dei prodotti offerti: sono frequenti accordi di marketing tra imprese che, pur non operando negli stessi settori, dispongono di risorse complementari. (Es. Mc Donald’s e Agip).
	\item Economie di costo: attraverso la cooperazione nelle diverse aree operative e strategiche del marketing, le imprese possono ridurre i costi legati ad iniziative promozionali, alla distribuzione dei prodotti, all’accesso ai media, all’acquisizione di competenze necessarie alla commercializzazione di un nuovo prodotto.
	\item Accrescere le risorse immateriali: qualsiasi collaborazione è in grado di arricchire il patrimonio immateriale delle imprese partner, attraverso la condivisione di esperienze, competenze e disponibilità. Di conseguenza, può permettere alle singole imprese partner di guadagnare posizioni nella continua ricerca del vantaggio competitivo.
\end{itemize}
